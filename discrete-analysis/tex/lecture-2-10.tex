\documentclass[document.tex]{subfiles}

\begin{document}
\section{Теорема Гливенко-Кантелли}
Пусть $\xi_1, \xi_2, \cdots$ -- независимые одинкаово распределенные случайные величины. Пусть $F(x)$ -- функция
распределения. Эмперическая функция распределения -- 
\[
    \hat F_n(x_1, \cdots, x_n; x) = \frac{1}{n}\sum_{i = 1}^{n} I \{x_i \leq x \}
\]
\begin{theorem}[Гливенко-Кантелли]
\[
    P(\sup_{x} |\hat F(\xi_1, \cdots, \xi_n; x) - F(x)| \rightarrow 0) = 1
\]
\end{theorem}

УЗБЧ для схемы испытаний Бернулли:
Пусть $\eta_1, \cdots, \eta_n$ -- случайные величины, причём
\[
    \eta_i = \begin{cases}
        1, p \\
        0, q
    \end{cases}
\]
Тогда $\overline X \cae p$

\[
    \{A_1^x, \cdots, A_n^x, \cdots\}_{x \in \mathcal{X}}
\]
Пример:
\[
    \{\{\xi_1 \leq x\}, \{\xi_2 \leq x\}, \cdots\}_{x \in \mathcal{X}}
\], причем $\forall x \in \mathcal{X}:$ $\{A_i^x : i \in \{1, 2, \cdots\}\}$ -- независимы в совокупности, причем
$P(A_i^x) = p^x$
Вопрос Вапника и Червоненкиса: при каких условиях 
\[
    P(\sup_{x \in \mathcal{X}} | \frac{I_{A_1^x} + I_{A_2^x} + \cdots}{n} - p^x| \rightarrow 0) = 1
\]
Предположим, что ранжированное пространство $(\Omega, \{A_i^x\}_{i, x \in \mathcal{X}})$ имеет конечную размерность
Вапника Червоненкиса. Тогда равномерная сходимость в УЗБЧ выполнена. Иначе нет.

\section{Матрица Адамара}
\[
    A = (a_{ij}), a_{ij} \in \{-1, 1\}
\]
Строчки попарно ортогональны.
Упражнение: Определение матрицы Адамара корректно, если заменить на то, что столбцы попарно ортагональны.

\begin{statement}
    Любая матрица Адамара без ограничения общности (с точностью до домножения строчек или столбцов на -1)
    имеет первую строчку и первый столбец из одних единиц. Это называется
    нормальной формой матрицы Адамара, обозначается $H$
\end{statement}
При $n > 1$ $n$ обязательно чётное. Так как вторая строчка ортоганальна первой, то $a_{21} + \cdots + a_{an} = 0$

Более того, если $n > 2$, то $4 | n$.

Гипотеза Адамара: $\forall n > 2, n \equiv 0 (4)$ существует матрица Адамара.

Пусть $n \equiv 0 (4)$. Рассмотрим граф $G(n, \frac{n}{2}, \frac{n}{4})$. $V = \{A \subset \{1, 2, \cdots, n\}, |A| =
\frac{n}{2} \}, E = \{(A, B): |A \cap B| = \frac{n}{4}\}$. Это граф изоморфен $G = (V', E')$, где $V' = \{x = (x_1,
    \cdots, x_n) : x_i \in \{-1, 1\}, x_1 + \cdots + x_n = 0\}, E' = \{\{x, y\}, (x, y) = 0\}$. $n - 1$ клика в графе
    $G'$ -- это все строчки матрицы Адамара без первой.

\begin{theorem}
    $\forall \varepsilon > 0: \exists n_0: \forall n \geq n_0:$ среди $n$ и $n(1 + \varepsilon)$ есть порядок матрицы
    Адамара.
\end{theorem}

Можно попытаться строить вот так:

$1, 1, \cdots, 1, 1$

$1, 1, \cdots, 1, -1, -1, \cdots, -1$

$1, 1, \cdots, 1, -1, -1, \cdots, -1, 1, 1, \cdots, 1, -1, -1, \cdots, -1$

Пример при $n = 4$
\[
    \begin{vmatrix}
        1 & 1 & 1 & 1 \\
        1 & -1 & 1 & -1 \\
        1 & 1 & -1 & -1 \\
        1 & -1 & -1 & 1
    \end{vmatrix}
\]

Кронекеровское умножение матриц:
\[
    A * B = 
    \begin{bmatrix}
        a_{11}B & a_{12}B & \cdots & a_{1n}B \\
        \cdots \\
        a_{n1}B & a_{n2}B & \cdots & a_{nn}B \\
    \end{bmatrix}
\]
Упражнение: если $A, B$ -- матрицы Адамара, то $A * B$ -- матрица Адамара.
\end{document}

