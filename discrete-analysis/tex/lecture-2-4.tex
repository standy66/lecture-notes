\documentclass[document.tex]{subfiles}

\begin{document}
\section{Двудольные числа Рамсея}
\begin{definition}
    $b(k, k)$ -- это минимальное такое $l$, что при любой раскраске ребёр $K_{l, l}$ в красный и синий цвета, найдется
    одноцветный $K_{k, k}$ 
\end{definition}

\begin{theorem}
    \[
        b(k, k) \geq (1 + o(1))\frac{2}{e}k2^{k/2}
    \]
\end{theorem}

\begin{remark}
    Берём случайную раскраску ребёр полного двудольного графа $K_{l, l}$. Рассматриваем случайную величину $\xi = $
    число одноцветных $K_{k, k}$. $E \xi = \binom{l}{k}^2 \cdot 2^{1 - k^2}$. Это матожидание отличается от аналогичного
    для чисел Рамсея совсем чуть-чуть.
\end{remark}

\begin{theorem}[Конлон]
    \[
        b(k, k) \leq (1 + o(1))\log_2 k \cdot 2^{k+1}
    \]
\end{theorem}

\begin{lemma}
    Пусть числа $m, n, r, s \in \mathbb{N}$ и $p \in [0, 1]$ таковы, что $(s - 1) \binom{m}{r} < n \binom{mp}{r}$. Пусть
    $G_{m, n}$ -- любой подграф $K_{m, n}: \frac{|E(G_{m, n})|}{mn} \geq p$. Тогда в $G_{m, n}$ есть $K_{r, s}$
\end{lemma}

\begin{proof}
    Предположим противное. Пусть в $G_{m, n}$ нет $K_{r, s}$.

    Подсчитаем двумя разными способами число подграфов $K_{r, 1}$ в графе $G_{m, n}$.

    Первый способ соответствует предположению противного.
    $\binom{m}{r} \cdot (s - 1)$ -- максимальное количество $K_{r, 1}$ в $G_{m, n}$ в виду сделанного нами предположения
    противного.

    С другой стороны, обозначим $d_1, \cdots, d_n$ -- степени вершин графа $G_{m, n}$ в правой доле.
    Тогда количество таких $K_{r, 1}$ -- это 
    $\binom{d_1}{r} + \cdots + \binom{d_n}{r} \geq \binom{\frac{d_1 + \cdots + d_n}{n}}{r}$
    По условию это больше либо равно $n \binom{pm}{r}$. Пришли к противоречию.
\end{proof}

Мы знаем, что если $r^2 = o(m)$, то $\binom{m}{r} \sim \frac{m^r}{r!}$, $\binom{mp}{r} \sim \frac{(mp)^r}{r!}$

\begin{lemma}[Та же самая, только в асимптотическом виде]
    Пусть $m = m(k), n = n(k), r = r(k), s = s(k)$. $p \in [0, 1]$. Предположим, что $r^2 = o(m)$, $n > (s - 1) \cdot
    p^{-r} (1 + o(1))$. Пусть для каждого $k$ $G_{m, n}$ -- любой произвольный подграф $K_{m, n}$ такой, что
    $|E(G_{m, n})| \geq pmn$. Тогда в $G_{m, n}$ есть $K_{r, s}$
\end{lemma}

\begin{proof}[Докажем теорему Колона]
    Мы хотим доказать, что для $\varepsilon > 0$
    $b(k, k) \leq (1 + \varepsilon) (\log_2 k) \cdot 2^{k+1}, k \geq k_0$
    Обозначим $l = (1 + \varepsilon) \cdot (\log_2 k) \cdot 2^{k+1}$. Это равносильно тому, что при любой раскраске
    рёбер графа $K_{l, l}$ в красный и синий цвета найдется одноцветный $K_{k, k}$. Зафиксируем произвольную раскраску.
    Назовём вершину красной, если её красная степень не меньше, чем синяя. В противном случая назовём её синей.
    Б.о.о считаем, что в правой доле красных хотя бы $\frac{l}{2}$. Возьмём из них первые $\frac{l}{2}$. Пусть $m(k) =
    l(k), n(k) = \frac{l(k)}{2}$. $G(m, n)$ -- это граф из красных рёбер. $p = \frac{1}{2}$. Положим $s(k) = k^2 \log_2
    k$, $r(k) = k - 2\log_2 k$
    

    $\frac{l}{2} = (1 + \varepsilon)(\log_2 k) \cdot 2^{k} > (k^2 \log_2 k - 1) 2^{k - 2\log_2 k}$
    
    Тогда из леммы следует, что при каждом $k \geq k_0$ в $G_{m, n}$ есть $K_{r, s}$
    Рассмотрим $m = k^2 \log_2 k$, $n = l - (k - 2\log_2 k)$ Возьмём $G_{m, n}$ -- из красных рёбер. $r = k, s = 2\log_2
    k$, $p = (\frac{l}{2} - k) / l = \frac{1}{2} - \frac{k}{l}$ После второго применения леммы победа
\end{proof}

\end{document}

