\documentclass[document.tex]{subfiles}

\begin{document}
\section{Числа Рамсея}
\begin{definition}
	Число Рамсея $R(s, t)$ для натуральных $s$ и $t$ -- это минимальное натуральное число $n$, такое, что при любой реберной раскарске полного графа на $n$ вершинах в два цвета, либо найдется полный подграф на $s$ вершинах первого цвета, либо полный подграф на $t$ вершинах второго цвета.
\end{definition}

\begin{definition}[Числа Рамсея, альтернативное определение]
	$R(s, t)$ -- минимальное такое $n$, что для любого графа на $n$ вершинах в нем есть либо $K_s$ клика, либо $\overline K_t$ антиклика
\end{definition}
\begin{example}
	~\begin{enumerate}
		\item $R(3, 3) = 6$
		\item $R(1, t) = 1$
		\item $R(2, t) = t$
	\end{enumerate}
\end{example}

\begin{statement}
	\[(
		\frac{1}{4} + o(1))\frac{t^2}{\ln t} \leq R(3, t) \leq (1 + o(1))\frac{t^2}{\ln t}
	\]
\end{statement}

\begin{remark}
	1/4 была получена в 2013 году, а 1/162 -- Кимом. Числа Рамсея были придуманы Рамсеем в 1930 году. В 1935 году Эрдеш и Секереш переоткрыли их в своей работе.
\end{remark}

\begin{statement}
	\[
		R(3, t) \geq c \frac{t^2}{\ln^2 t}
	\]
\end{statement}

\begin{theorem}
	$R(s, t) \leq R(s-1, t) + R(s, t-1)$
\end{theorem}

\begin{proof}
	Обозначим $r_1 = R(s - 1, t), r_2 = R(s, t - 1), n = r_1 + r_2$. Положим также $\Deg+ v = \Deg{} v$, $\Deg- v = n - 1 - \Deg+ v$.

	Рассмотрим граф $G$ на $n$ вершинах и произвольную вершину $v$ этого графа. Ясно, что либо $\Deg+ v \geq r_1$, либо $\Deg- v \geq r_2$. В первом случае вершина $v$ смежна с подграфом на $r_1$ вершинах, в котором есть либо $\overline K_t$ (в этом случае все хорошо), либо $K_{s - 1}$. Но тогда этот $K_{s - 1}$ вместе с вершиной $v$ дает $K_s$ и тоже все хорошо. Второй случай рассматривается аналогично.
\end{proof}

\begin{corollary}
	\[
		R(s, t) \leq \binom{s - 1}{s + t - 2}
	\]
\end{corollary}

\begin{proof}
	Индукция по $s + t$: применяем рекурсивную формулу из прошлой теоремы, а также рекурсивную формулу для треугольника Паскаля.
\end{proof}

\begin{definition}
	Диагональные числа Рамсея -- это числа $R(s, s)$.
\end{definition}

\begin{corollary}[из следствия]
	\[
		R(s, s) \leq \binom{s - 1}{s + s -2} \approx \frac{4^s}{\sqrt{\pi s}}
	\]
\end{corollary}


\begin{remark}
	%``Прошло восемьдясят лет (надо отметить, а у меня только минералка с собой), а гора, в каком то смысле, родила лишь мышь''.
    Самая сильная верхняя оценка, которую людям удалось доказать -- это $$\exists \gamma > 0 : R(s, s) \leq 4^s \cdot e^{-\gamma \cdot \frac{\ln^2
    s}{\ln \ln s}}$$. Это сделал Конлон. 
    
    %Эрдеш очень любил придумавать задачи. Мало того, что он их придумывал, он ещё и цены задавал. ``За решение этой задачи a бы дал 300 долларов, за эту -- 500, в вот за эту -- 3000, это же целая проблема''. Правда сам он ничего давать не мог, у него денег не было. Но у него были последователи, у которых они были. Например, у него был Р. Грэхем. Он лет пятьдесят назад уподобился Эрдешу и сказал: я бы дал 100 долларов тому, кто заменит вот эту четверку на что-то меньшее. Но 100 долларов с тех пор так никто и не получил.
\end{remark}

\begin{theorem}
	Пусть дано $s$ -- натуральное. Найдем такое $n$, что
	\[
		\binom{n}{s} \cdot 2^{1-\binom{s}{2}} < 1
	\] Тогда $R(s, s) > n$
\end{theorem}

\begin{proof}
	Докажем эту задачу вероятностным методом. То, что нужно доказать, равносильно тому, что $\exists$ раскраска ребер полного графа на $n$ вершинах при которой нет одноцветной клики на $s$ вершинах.

	Рассмотрим вероятностное пространство $G(n, \frac{1}{2})$.
	Введем случайную величину $\xi$ -- количество одноцветных $s$-клик. Пусть $\xi_S$ -- другая индикаторная случайная величина, которая равна 1, если подграф $S$ одноцветен. Тогда
	\[
		P(\xi_S = 1) = 2 \cdot \left(\frac{1}{2}\right)^{\binom{s}{2}} = 2^{1-\binom{s}{2}}
	\]
	Но вследствие линейности математического ожидания:
	\[
		\xi = \sum_{S, |S| = s} \xi_S
	\]
	А значит,
	\[
		E\xi = \sum_{S, |S| = s} E\xi_S = \binom{n}{s} \cdot 2^{1-\binom{s}{2}} < 1
	\]
	А значит существует такая раскраска, что $\xi = 0$
\end{proof}

\begin{corollary}
	$$R(s, s) \geq (1+o(1))\frac{s}{e\sqrt{2}} 2^{s/2}$$
\end{corollary}
\begin{proof}
	Положим $n = (1+f(s)) \frac{s}{e\sqrt{2}}2^{s/2}$, где $f(s) = o(1)$
	$$C_n^s \cdot 2^{1 - C_s^2} \leq \frac{n^s}{s!} \cdot 2^{1 - \frac{s(s-1)}{2}} = (1+f(s))^s \frac{s^s}{e^s 2^{s/2}s!} \cdot 2^{s^2/2} \cdot 2^{1 - s^2/2 + s/2}$$
	$$= \frac{(1 + o(1))^s \cdot 2}{(1+o(1)) \sqrt{2 \pi s}} < 1$$
	при правильтном выборе $f(s)$

\end{proof}

\begin{theorem}[Эрдеша]
	$$R(s, s) \geq (1+o(1)) \frac{s}{e} \cdot 2^{s/2}$$
\end{theorem}

\begin{theorem}[Спенсера]
	$$R(s, s) \geq (1+o(1)) \cdot \frac{s\sqrt{2}}{e} 2^{s/2}$$

\end{theorem}

\begin{definition}
	Событие $B$ не зависит от совокупности событий $A_1, \ldots, A_n$, если $$\forall J \subset \{1, \ldots, n\}: P(A | \cap_{j \in J} A_j) = P(B)$$
\end{definition}

\begin{lemma}[симметричная локальная лемма Ловаса]
	Пусть $A_1, \ldots, A_n$ -- события. Путь дополнительно $\exists p : \forall i : P(A_i) \leq p$. Дополнительно предположим, что $\forall i : A_i$ не зависит от совокупности всех остальных событий, кроме не более чем $d$ штук. Пусть также $ep(d+1) \leq 1$. Тогда $P(\cap_{i=1}^nA_i^c) > 0$
\end{lemma}

\begin{proof}[Доказательство теоремы Спенспера]
	Нам нужно доказать, что $$P(\cap_{S, |S| = s} A_S) > 0$$. $P(A_s) = 2^{1 - C_s^2} = p$. Чему же равно $d$? $A_S$ зависит от тех $A_T$, у которых $|S \cap T| \geq 2$. Тогда $d \leq C_s^2 \cdot C_{n-2}^{s-2}$. Осталось доказать, что $ep(d+1) < 1$.

	\begin{multline*}
		ep(d+1) = e \cdot 2^{1 - C_s^2} \cdot (C_s^2 C_{n-2}^{s-2} - 1) = e(1+o(1))2^{1 - C_s^2}C_s^2C_{n-2}^{s-2} \\
		\leq e (1+o(1)) 2^{1 - s^2/2 + s/2} \cdot \frac{s^2}{2} \cdot \frac{n^{s-2}}{(s-2)!} = e (1+o(1)) 2^{1 - s^2/2 + s/2} \cdot \frac{s^4}{2} \cdot \frac{(1+o(1))^{s-2} 2^{s/2 - 1} }{e^{s-2}(s)!}s^{s-2}\cdot 2^{s^2/2-s} \\
		\frac{(1+o(1))es^{s+2}(1+o(1))^{s-2}}{e^{s-2}2(1+o(1))\sqrt{2 \pi s} \left(\frac{s}{e}\right)^{s}}
	\end{multline*}
Если взять $o(1) = -\frac{1}{\sqrt{s}}$, то все получится.
\end{proof}

\end{document}

