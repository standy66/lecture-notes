\documentclass{article}

%\usepackage{subfiles}
\usepackage{../../mystyle}

\begin{document}
\section{Определения}
\begin{definition}
    Пусть $F$ -- факториальное кольцо, $f(x) \in F[x]$. Тогда содержание $c(f)$ многочлена $f(x)$ -- это наибольшый
    общий делитель его коэффициентов.
\end{definition}

\begin{definition}
    Пусть $F$ -- факториальное кольцо, $f(x) \in F[x]$. Тогда $f(x)$ -- примитивный, если его содержание тривиально:
    $c(f) \sim 1$
\end{definition}

\begin{definition}
    Характеристика поля $F$ -- $char F$ -- это минимальное количество единиц, сумма которых равна нулю. Если никакя
    сумма единиц не дает нуля, то говорят, что $char F = 0$, иначе $F$ -- поле конечное характеристики.
\end{definition}

\begin{definition}
    Пусть $F$ -- поле. Тогда $K$ называется расширением $F$, если $K$ -- поле, и $F \subset K$
\end{definition}

\begin{definition}
    Пусть $K$ -- расширение $F$. Элемент $\alpha \in K$ называется алгебраическим над $F$, если $\exists f(x) \in F[x]:
    f(\alpha) = 0$
\end{definition}

\begin{definition}
    Пусть $K$ -- расширение $F$. Элемент $\alpha \in K$ называется трансцендентным над $F$, если $\forall f(x) \in F[x]:
    f(\alpha) \neq 0$
\end{definition}

\begin{definition}
    Расширение $K$ называется алгебраическим, если все его элементы алгебраические.
\end{definition}

\begin{example}
    $\mathbb{C}$ -- алгебраическое расширение $\mathbb{R}$, так как любой его элемент $z = a + ib$ является корнем
    многочлена $\left( \frac{x - a}{b} \right)^2 + 1 = 0$.
\end{example}

\begin{example}
    $\mathbb{R}$ -- не алгебраическое расширение $\mathbb{Q}$, так как $\pi$ не является корнем ни какого многочлена над
    $\mathbb{Q}$.
\end{example}

\begin{definition}
    Пусть $K$ -- расширение над $F$. $\alpha \in K$ -- алгебраический элемент. Тогда $m_{\alpha}(x)$ -- это минимальный
    (по степени) многочлен со старшим коэфициентом, равным единице, который обнуляет $\alpha$.
\end{definition}

\begin{definition}
    Пусть $K \supset F$ -- расширение, $\gamma \in K$. $F(\gamma)$ -- минимальное по включению поле, содержащее $F$ и
    $\gamma$.
\end{definition}

\begin{definition}
    Пусть $F$ -- поле, $f(x) \in F[x]$. Тогда поле разложения многочлена $f(x)$ -- это такое минимальное по включению
    расширение $K \supset F$, что $f$ раскладывается на множители над $K$.
\end{definition}

\begin{definition}
    Поле $K$, удовлетворяющее любому из следующих эквивалентных условий, называется алгебраически замкнутым:
    \begin{enumerate}
        \item Любой многочлен $f(x) \in K[x]$ ненулевой степени имеет корень в $K$.
        \item Полем разложения любого многочлена $f(x) \in K[x]$ является $K$
        \item Любой неприводимый многочлен $f(x) \in K[x]$ имеет степень 1.
        \item Любое алгебраическое расширение $L \supset K$ тривиально, то есть $L = K$.
    \end{enumerate}
\end{definition}

\begin{definition}
    Пусть $F$ -- поле, тогда его алгебраическим замыканием $\overline F$ называется такое его алгебраическое расширение,
    которое является алгебраически замкнутым.
\end{definition}

\begin{definition}
    Расширение $K \supset F$ называется сепарабельным, если все его элементы $\alpha \in K$ сепарабельны над $F$.
\end{definition}

\begin{definition}
    Элемент $\alpha$ расширения $K \supset F$ называется сепарабельным, если его минимальный многочлен $m_{\alpha}$ не
    содержит кратных корней.
\end{definition}

\begin{definition}
    Назовём расширение $K \supset F$ радикальным, если существует башня расширений $K \supset K_n \supset K_{n - 1}
    \supset \cdots \supset K_1 \supset F$, где $K_1 = F(\alpha_0), K_2 = K_1(\alpha_1), \cdots, K_n = K_{n
    -1}(\alpha_{n - 1}), K = K_n(\alpha_n)$, $\alpha_i$ -- корень многочлена $x^{n_i} - a_i = 0$, $a_i \in K_i$.

    Многочлен $f(x) \in F[x]$ называется разрешимым в радикалах, если его поле разложения является радикальным
    расширением $F$.
\end{definition}

\begin{definition}
    $\xi_n$ -- примитивный корень $n$-ной степени из единицы -- это такой элемент $F$, что $\xi_n^n = 1$, а степени
    $\xi_n$ -- это всевозможные корни $n$-ной степени из единицы.
\end{definition}

\begin{definition}
    Два элемента называются сопряженными, если их минимальные многочлены совпадают.
\end{definition}

\begin{statement}[критерий неприводимости Эзенштейна]
    Для многочлена $f(x) = a_n x^n + \cdots + a_0 \in F[x]$ он является неприводимым, если существет такое простое $p$,
    что $p \not | a_n$, $p | a_i$, $i \in \{n - 1, \cdots, 0\}$, $p^2 \not | a_0$
\end{statement}

\begin{definition}
    Конечное расширение $K \supset F$ называется нормальным, если оно удовлетворяет одному из следующих равносильных
    условий:
    \begin{enumerate}
        \item $\forall \alpha \in K: \forall \beta$ -- сопряженный к $\alpha$, $\beta \in K$
        \item $K$ -- поле разложения некоторого семейства многочленов.
        \item $\forall \varphi: K \mapsto \overline F$ -- изоморфизм, сохраняющий $F$, также является автоморфизмом $K$.
    \end{enumerate}
\end{definition}

\begin{definition}
    Пусть $G = Aut_F K$ -- группа автоморфизмов конечного расширения $K \supset F$, сохраняющих $F$. $\forall H < G$
    рассмотрим $K^H = \{x \in K: \forall h \in H: hx = x\}$. Тогда конечное сепарабельное расширение $K \supset F$
    называется расширением Галуа, если для него выполнено одно из следующих эквивалентных условий:
    \begin{enumerate}
        \item $K$ -- нормальное расширение
        \item $[K:F] = |G|$
        \item $K^G = F$
    \end{enumerate}
\end{definition}

\begin{definition}
    Группа Галуа расширения $K \supset F$ -- это группа автоморфизмов этого расширения $K$, сохраняющих $F$.
\end{definition}

\section{Вопросы}
\begin{statement}
    Пусть $f(x) = a_n x^n + \cdots + a_0$ -- многочлен над $\mathbb{Z}[x]$. Пусть также $\exists p$ -- простое,
    что $p$ делит все $a_i$, кроме первого, а $a_0$ не делится на $p^2$. Тогда $f(x)$ -- неприводимый.
\end{statement}

\begin{proof}
    Пусть не так, тогда $\exists g(x) = b_m x^m + \cdots + b_0, h(x) = c^l x^l + \cdots + c_0 \in \mathbb{Z}[x]$, причем
    $f = gh$, $m < n, l < n$. Тогда $a_i = \sum_{(j, k): j + k = i} b_j c_k$. Известно, что $p | a_0 = b_0 \cdot c_0$.
    Тогда либо $p | b_0$, либо $p | c_0$. Причем, поскольку $\neg p^2 | a_0$, то $c_0$ либо $b_0$ соответсвенно не
    делится на $p$. Пусть без ограничения общности $p | b_0, \neg p | c_0$. Тогда из того, что $p | a_1 = b_1 a_0 + b_0
    a_1$. Тогда $b_1$ делится на $p$. Аналогично для всех остальных $b_i$. Тогда $p | b_m$, следовательно $a_n = b_m
    c_k$ также делится на $p$. Противоречие.
\end{proof}

\begin{statement}
    Многочлен $\Phi_p(x) = x^{p - 1} + x^{p - 2} + \cdots + x + 1$, где $p$ -- простое число, неприводим.
\end{statement}

\begin{proof}
    $\Phi_p(x) = \frac{x^p - 1}{x - 1}$. Сделаем замену $t = x - 1$. Тогда $\Phi_p = \frac{(t + 1)^p - 1}{t} = \sum_{i =
    1}^p \binom{p}{i} t^{i - 1} = t^{p - 1} + \sum_{i = 2}^{p - 1} \binom{p}{i} t^{i - 1} + p$ -- неприводим по критерию
    Эйзенштейна.
\end{proof}

\begin{statement}
    Пусть $F$ -- факториальное кольцо. Тогда неразложимые многочлены $F[x]$ степени 0 -- это в точности простые элементы
    $F$.
\end{statement}

\begin{proof}
    Пусть $p$ -- неразложимый в $F[x]$. Тогда $\forall a, b, p = ab:$ без ограничения общности $a \in F[x]^*$. Но $a, b$
    -- тоже многочлены степени 0, то есть константы. Но тогда $a \in F^*$. Значит, $p$ -- неразложимый, а следовательно,
    простой в $F$.

    Пусть наоборот, $p$ -- простой элемент $F$. Пусть он разложим над $F[x]$, то есть $\exists f, g \not \in F[x]^*: fg
    = p$. Тогда степень $f$ и $g$ -- ноль. Тогда $p = fg$ над $F$. То есть $p$ -- не простой.
\end{proof}

\begin{statement}
    Для произвольных многочленов $f, g \in F[x]$ над факториальным кольцом $F$ выполнено равенство $c(fg) = c(f)c(g)$.
\end{statement}

\begin{proof}
    Пусть $fg = c(f) \hat f c(g) \hat g$, где $\hat f, \hat g$ -- примитивные. Докажем, что $\hat f \hat g$ -- тоже
    примитивный. Пусть $\hat f \hat g = a^n x^n + \cdots + a_0, \hat f = b_m x^m + \cdots + b_0, \hat g = c_l x^l +
    \cdots + c_0$. Тогда $a_k = \sum_{i = 0}^k b_i c_{l - i}$. Предположим, что $\hat f \hat g$ -- не примитивный. Тогда
    $\exists p: p | c(\hat f \hat g)$, $p$ -- простое.
\end{proof}

\begin{statement}
    Любое конечное поле имеет отличную от нуля характеристику.
\end{statement}

\begin{proof}
    Пусть не так, тогда $char F = 0$. Тогда никакая сумма единиц не обращается в ноль. Рассмотрим следующие элементы
    поля: $1, 2 := 1 + 1, 3 := 1 + 1 + 1, 4 := 1 + 1 + 1 + 1$. Они все различны, ведь если не так, скажем, $i = j$, то
    $0 = i - j = 1 + 1 + \cdots + 1$. Тогда $char F \neq 0$. Но их счетное число, и они все различны. Значит, поле как
    минимум счётно.
\end{proof}

\begin{statement}
    Любой нетривиальный гомоморфизм полей является инъективным.
\end{statement}

\begin{proof}
    Пусть $\varphi: F \mapsto K$ -- гомоморфизм, нетривиальный. Его ядро -- это идеал в $F$. Но $F$ -- поле, а там нет
    нетривиальных идеалов. Значит, $Ker \varphi = F$, либо $Ker \varphi = 0$. Но первое не выполнено, так как
    гомоморфизм нетривиальный. Но тогда $Ker \varphi = 0$, то есть гомоморфизм $\varphi$ инъективен.
\end{proof}

\begin{statement}
    Если $char F \neq 0$, то $char F$ -- простое число
\end{statement}

\begin{proof}
    Пусть $char F = mn$, $m > 1, n > 1$.
    Обозначим $mn := 1 + 1 + \cdots + 1$ ($mn$ раз), $m := 1 + 1 + \cdots + 1$ ($m$ раз), $n := 1 +
    1 + \cdots + 1$ ($n$ раз). Тогда, воспользовавшись дистрибутивностью, имеем: $mn = 1 + 1 + \cdots + 1 = (1 + 1 +
    \cdots + 1) \cdot (1 + 1 + \cdots + 1) = m \cdot n = 0$. Поскольку делителей нуля в поле нет, то либо $m = 0$, либо
    $n = 0$. Но тогда $char F \leq m$ или $n$ соответсвенно. Противоречие.
\end{proof}

\begin{statement}
    Любое поле $F$ нулевой характеристики имеет $\mathbb{Q}$ в качестве подполя.
\end{statement}

\begin{proof}
    В $F$ есть что-то, изоморфное $\mathbb{Z}$, значит есть что-то изоморфное $Quot \mathbb{Z} = \mathbb{Q}$, так как
    это поле.
\end{proof}

\begin{statement}
    Если существует нетривиальный гомоморфизм полей $\varphi: F \mapsto K$, то $char F = char K$
\end{statement}

\begin{proof}
    Рассмотрим $1_F, 1_F + 1_F, \cdots$ в $F$. Под действием гомоморфизма (который является инъекцией по предыдущему
    утверждению) эти суммы единиц переходят в различные суммы единиц в $K$. Значит, $char F \leq char K$. Но если $0 =
    1_F
    + 1_F + \cdots + 1_F$, то $\varphi(0) = 0 = \varphi(1_F) + \cdots + \varphi(1_F) = 1_K + \cdots + 1_K$. Тогда $char
    K \leq char F$
\end{proof}

\begin{statement}
    Расширение поля является линейным пространством над этим полем.
\end{statement}

\begin{proof}
    Просто проверяем свойства линейных пространств.
\end{proof}

\begin{statement}
    Пусть $f(x)$ -- неприводимый многочлен степени $n$. $K = F[x] / (f(x))$. Тогда многочлен $f(x)$ имеет корень в $K$.
\end{statement}

\begin{proof}
    Обозначим за $[a] := a + (f(x))$ -- класс эквивалентности элемента $a$ в факторе $K$.
    Тогда $f([x]) = [f(x)]$, так как достаточно выбрать
    представителя в $[x]$, провести все операции над ним, и потом взять от полученного выражения класс эквивалентности.
    Но $[f(x)] = (f(x)) = 0 + (f(x)) = 0_K$. Тогда $[x]$ -- корень многочлена $f$ в $K$.
\end{proof}

\begin{statement}
    Пусть $f(x)$ -- неприводимый многочлен степени $n$. $K = F[x] / (f(x))$. Тогда степень расширения $[K : F] = n$.
\end{statement}

\begin{proof}
    $[1], [x], [x^2], \cdots, [x^{n - 1}]$ является базисом в $K$. Действительно, пусть $[a] = a + (f(x)) \in K$.
    Разделим $a(x)$ на $f(x)$ с остатком: $a(x) = f(x)g(x) + c(x)$, причём $\deg c < n$. Тогда $[a] = c(x) + f(x)g(x) +
    (f(x)) = c(x) + (f(x))$. Пусть $c(x) = c_m x^m + \cdots + c_0$. Тогда $[a] = c^m [x^m] + \cdots + c_0 \cdot [1]$.
    Значит, любой элемент $K$ выражается через $[1], [x], \cdots$. Докажем, что они ЛНЗ. Пусть не так, тогда $\exists
    c_{n - 1}, \cdots, c_0: c_{n - 1}[x^{n - 1}] + \cdots + c_0 [1] = 0$. Тогда $[c_{n - 1} x^{n - 1} + \cdots + c_0] =
    0$. То есть многочлен степени $n - 1$ лежит в $(f(x))$, а такого быть не может, ведь в $(f(x))$ могут лежать только
    многочлены, кратные $f(x)$.
\end{proof}

\begin{statement}
    Для любого $g(x) \in F[x]$ найдется такое расширение $K \supset F$, что $g(x)$ имеет корень в $K$.
\end{statement}

\begin{proof}
    Разложим $g(x)$ на неприводимые. Пусть $h(x)$ -- один из неприводимых в разложении $g(x)$. Тогда по предыдущему
    пункту $F[x] / (h(x))$ -- это поле, содержащее $F$, причём $h(x)$, а следовательно, $g(x)$ содержат в нём корень.
\end{proof}

\begin{statement}
    $[K : F] = [K : L] \cdot [L : F]$
\end{statement}

\begin{proof}
Пусть $g_1, \cdots g_m$ -- базис в $L$ над $F$. $e_1, \cdots, e_n$ -- базис $K$ над $L$. Тогда $g_ie_j$ -- базис $K$ над
$F$. Действительно, пусть $k \in K. k = \sum_{i = 1}^m a_i g_i$, $a_i \in L$. Пусть $a_i = \sum_{j = 1}^n b_{ij}e_j$.
Тогда $k = \sum_{i = 1}^m \sum_{j = 1}^n b_{ij}e_j g_i$. Значит, любой элемент $K$ выражается как линейная комбинация
базиса.

Теперь докажем, что $g_ie_j$ -- ЛНЗ. Пусть нет, тогда $\exists b_{ij}: \sum_{i = 1}^m \sum_{j = 1}^n b_{ij} e_i g_j =
0$. Вынося $e_i$ за скобки, имеем $\sum_{i = 1}^m \left( \sum_{j =1 }^n b_{ij}g_j \right) e_i = 0$. Понятно, что каждое
из коэффициентов при $e_i$ не ранво нулю, так как $g_j$ -- базис. Но тогда мы придем к противоречию с тем, что $e_i$ --
базис.
\end{proof}

\begin{statement}
    Любое конечное расширение является алгебраическим.
\end{statement}

\begin{proof}
    Расширение конечное, значит, в нём можно выделить базис размера, скажем $n$. Тогда элементы $1, \alpha, \alpha^2,
    \cdots, \alpha^n$ всегда будут являтся линейно зависимыми, так как их $n + 1$ штука. Но тогда существует многочлен,
    обнуляющий $\alpha$.
\end{proof}

\begin{statement}
    У любого поля существует не алгебраическое расширение -- $Quot F[x]$
\end{statement}

\begin{proof}
    $Quot F[x]$ действительно является полем и содержит $F$. Но, скажем, элемент $x \in Quot F[x]$ не является
    алгебраическим. Пусть это не так, тогда $\exists 0 \neq f(x) \in F[x]: f(x)$ при подстанвке $x$ и интерпретировании как
    выражения из $Quot F[x]$ будет равен нулю. Но такого быть не может по выбору $f(x)$.
\end{proof}

\begin{statement}
    $m_{\alpha}(x)$ -- неприводимый.
\end{statement}

\begin{proof}
    Пусть $m_{\alpha}(x) = f(x)g(x)$. Так как $F$ -- поле, а $m_{\alpha}$ -- минимальный для $\alpha$, то либо
    $f(\alpha) = 0$, либо $g(\alpha) = 0$. Но степени обоих $f, g$ меньше степени $m_{\alpha}$, значит $m_{\alpha}$ --
    не минимальный многочлен.
\end{proof}

\begin{statement}
    Минимальный многочлен делит все остальные обнуляющие многочлены.
\end{statement}

\begin{proof}
    Пусть не так, скажем, $f(\alpha) = 0$. Поделим $f$ на $m_{\alpha}$ с остатком. Тогда $f(x) = g(x) m_{\alpha}(x) +
    c(x)$, причём степень $c(x)$ меньше степени $m_{\alpha}$. Тогда $0 = f(\alpha) = g(\alpha) m_{\alpha}(\alpha) +
    c(\alpha)$, то есть $c(\alpha)$ -- минимальный многочлен меньшей степени, обнуляющий $\alpha$. Противоречие.
\end{proof}

\begin{statement}
    $m_{\alpha}$ -- единственный неприводимый многочлен со старшим коэффициентом $1$, у которого $\alpha$ является
    корнем.
\end{statement}

\begin{proof}
    Пусть $m_{\alpha}(x) = x^n + a_{n - 1} x^{n - 1} + \cdots + a_0$. Пусть $f(x)$ -- другой неприводимый многочлен со старшим
    коэфициентом $1$, у которго $\alpha$ является корнем. Тогда степень $f(x)$ не может превосходить степень
    $m_{\alpha}$, ведь $m_{\alpha} | f$. Но степень $f(x)$ не может быть меньше степени $m_{\alpha}$, ведь тогда
    $m_{\alpha}$ не минимальный. Итого, $f(x) = x^{n} + b_{n - 1}x^{n - 1} + \cdots + b_0$. Вычтем из $f(x)$
    $m_{\alpha}(x)$. Тогда $0 = f(\alpha) - m_{\alpha}(\alpha) = (a_{n -1 } - b_{n - 1}) \alpha^{n - 1} + \cdots + (a_0
    - b_0)$ -- многочлен меньшей степени, обнуляющий $\alpha$.
\end{proof}

\begin{statement}
    ~\begin{enumerate}
        \item $[\mathbb{Q}(\sqrt{2}) : \mathbb{Q}] = 2$
        \item $[\mathbb{Q}(2^{1/3}): \mathbb{Q}] = 3$
        \item $[\mathbb{Q}(\sqrt{2}i): \mathbb{Q}] = 2$
        \item $[\mathbb{Q}(\xi_5): \mathbb{Q}] = 4$
        \item $[\mathbb{Q}(\sqrt{2} + \sqrt{3}): \mathbb{Q}] = 4$
        \item $[\mathbb{Q}(\xi_8) : \mathbb{Q}] = 4$
    \end{enumerate}
\end{statement}

\begin{proof}
    \begin{enumerate}
        \item $x^2 - 2$ -- неприводим по критерию Эйзенштейна и обнуляет $\alpha$, значит, минимальный.
        \item $x^3 - 2$ -- аналогично
        \item $x^2 + 2$ -- аналогично
        \item $x^5 - 1 = (x - 1) (x^4 + x^3 + x^2 + x^1 + x)$. Второй множитель обнуляет $\xi_5$ и является неприводимым
            многочленом (было доказано ранее).
        \item $1, \sqrt{2}, \sqrt{3}, \sqrt{6}$ -- базис, посокльку ЛНЗ и через него выражаются $1, (\sqrt{2} +
            \sqrt{3}), (\sqrt{2} + \sqrt{3})^2, (\sqrt{2} + \sqrt{3})^3, \cdots$
        \item 
    \end{enumerate}
\end{proof}


\end{document}
