\documentclass{article}

%\usepackage{subfiles}
\usepackage{../../mystyle}

\begin{document}
\section{Определения}
\begin{definition}
    Пусть $F$ -- факториальное кольцо, $f(x) \in F[x]$. Тогда содержание $c(f)$ многочлена $f(x)$ -- это наибольшый
    общий делитель его коэффициентов.
\end{definition}

\begin{definition}
    Пусть $F$ -- факториальное кольцо, $f(x) \in F[x]$. Тогда $f(x)$ -- примитивный, если его содержание тривиально:
    $c(f) \sim 1$
\end{definition}

\begin{definition}
    Характеристика поля $F$ -- $char F$ -- это минимальное количество единиц, сумма которых равна нулю. Если никакя
    сумма единиц не дает нуля, то говорят, что $char F = 0$, иначе $F$ -- поле конечное характеристики.
\end{definition}

\begin{definition}
    Пусть $F$ -- поле. Тогда $K$ называется расширением $F$, если $K$ -- поле, и $F \subset K$
\end{definition}

\begin{definition}
    Пусть $K$ -- расширение $F$. Элемент $\alpha \in K$ называется алгебраическим над $F$, если $\exists f(x) \in F[x]:
    f(\alpha) = 0$
\end{definition}

\begin{definition}
    Пусть $K$ -- расширение $F$. Элемент $\alpha \in K$ называется трансцендентным над $F$, если $\forall f(x) \in F[x]:
    f(\alpha) \neq 0$
\end{definition}

\begin{definition}
    Расширение $K$ называется алгебраическим, если все его элементы алгебраические. 
\end{definition}

\begin{example}
    $\mathbb{C}$ -- алгебраическое расширение $\mathbb{R}$, так как любой его элемент $z = a + ib$ является корнем
    многочлена $\left( \frac{x - a}{b} \right)^2 + 1 = 0$.
\end{example}

\begin{example}
    $\mathbb{R}$ -- не алгебраическое расширение $\mathbb{Q}$, так как $\pi$ не является корнем ни какого многочлена над
    $\mathbb{Q}$.
\end{example}

\begin{definition}
    Пусть $K$ -- расширение над $F$. $\alpha \in K$ -- алгебраический элемент. Тогда $m_{\alpha}(x)$ -- это минимальный
    (по степени) многочлен со старшим коэфициентом, равным единице, который обнуляет $\alpha$.
\end{definition}

\begin{definition}
    Пусть $K \supset F$ -- расширение, $\gamma \in K$. $F(\gamma)$ -- минимальное по включению поле, содержащее $F$ и
    $\gamma$.
\end{definition}

\begin{definition}
    Пусть $F$ -- поле, $f(x) \in F[x]$. Тогда поле разложения многочлена $f(x)$ -- это такое минимальное по включению
    расширение $K \supset F$, что $f$ раскладывается на множители над $K$.
\end{definition}

\begin{definition}
    Поле $K$, удовлетворяющее любому из следующих эквивалентных условий, называется алгебраически замкнутым:
    \begin{enumerate}
        \item Любой многочлен $f(x) \in K[x]$ ненулевой степени имеет корень в $K$.
        \item Полем разложения любого многочлена $f(x) \in K[x]$ является $K$
        \item Любой неприводимый многочлен $f(x) \in K[x]$ имеет степень 1.
        \item Любое алгебраическое расширение $L \supset K$ тривиально, то есть $L = K$.
    \end{enumerate}
\end{definition}

\begin{definition}
    Пусть $F$ -- поле, тогда его алгебраическим замыканием $\overline F$ называется такое его алгебраическое расширение,
    которое является алгебраически замкнутым.
\end{definition}

\begin{definition}
    Расширение $K \supset F$ называется сепарабельным, если все его элементы $\alpha \in K$ сепарабельны над $F$.
\end{definition}

\begin{definition}
    Элемент $\alpha$ расширения $K \supset F$ называется сепарабельным, если его минимальный многочлен $m_{\alpha}$ не
    содержит кратных корней.
\end{definition}

\begin{definition}
    Назовём расширение $K \supset F$ радикальным, если существует башня расширений $K \supset K_n \supset K_{n - 1}
    \supset \cdots \supset K_1 \supset F$, где $K_1 = F(\alpha_0), K_2 = K_1(\alpha_1), \cdots, K_n = K_{n
    -1}(\alpha_{n - 1}), K = K_n(\alpha_n)$, $\alpha_i$ -- корень многочлена $x^{n_i} - a_i = 0$, $a_i \in K_i$.

    Многочлен $f(x) \in F[x]$ называется разрешимым в радикалах, если его поле разложения является радикальным
    расширением $F$.
\end{definition}

\begin{definition}
    $\xi_n$ -- примитивный корень $n$-ной степени из единицы -- это такой элемент $F$, что $\xi_n^n = 1$, а степени
    $\xi_n$ -- это всевозможные корни $n$-ной степени из единицы.
\end{definition}

\begin{definition}
    Два элемента называются сопряженными, если их минимальные многочлены совпадают.
\end{definition}

\begin{statement}[критерий неприводимости Эзенштейна]
    Для многочлена $f(x) = a_n x^n + \cdots + a_0 \in F[x]$ он является неприводимым, если существет такое простое $p$,
    что $p \not | a_n$, $p | a_i$, $i \in \{n - 1, \cdots, 0\}$, $p^2 \not | a_0$
\end{statement}

\begin{definition}
    Конечное расширение $K \supset F$ называется нормальным, если оно удовлетворяет одному из следующих равносильных
    условий:
    \begin{enumerate}
        \item $\forall \alpha \in K: \forall \beta$ -- сопряженный к $\alpha$, $\beta \in K$
        \item $K$ -- поле разложения некоторого семейства многочленов.
        \item $\forall \varphi: K \mapsto \overline F$ -- изоморфизм, сохраняющий $F$, также является автоморфизмом $K$.
    \end{enumerate}
\end{definition}

\begin{definition}
    Пусть $G = Aut_F K$ -- группа автоморфизмов конечного расширения $K \supset F$, сохраняющих $F$. $\forall H < G$
    рассмотрим $K^H = \{x \in K: \forall h \in H: hx = x\}$. Тогда конечное сепарабельное расширение $K \supset F$
    называется расширением Галуа, если для него выполнено одно из следующих эквивалентных условий:
    \begin{enumerate}
        \item $K$ -- нормальное расширение
        \item $[K:F] = |G|$
        \item $K^G = F$
    \end{enumerate}
\end{definition}

\begin{definition}
    Группа Галуа расширения $K \supset F$ -- это группа автоморфизмов этого расширения $K$, сохраняющих $F$.
\end{definition}

\section{Вопросы}
\begin{statement}
    Пусть $f(x) = a_n x^n + \cdots + a_0$ -- многочлен над $\mathbb{Z}[x]$. Пусть также $\exists p$ -- простое, 
    что $p$ делит все $a_i$, кроме первого, а $a_0$ не делится на $p^2$. Тогда $f(x)$ -- неприводимый.
\end{statement}

\begin{proof}
    Пусть не так, тогда $\exists g(x) = b_m x^m + \cdots + b_0, h(x) = c^l x^l + \cdots + c_0 \in \mathbb{Z}[x]$, причем
    $f = gh$, $m < n, l < n$. Тогда $a_i = \sum_{(j, k): j + k = i} b_j c_k$. Известно, что $p | a_0 = b_0 \cdot c_0$.
    Тогда либо $p | b_0$, либо $p | c_0$. Причем, поскольку $\neg p^2 | a_0$, то $c_0$ либо $b_0$ соответсвенно не
    делится на $p$. Пусть без ограничения общности $p | b_0, \neg p | c_0$. Тогда из того, что $p | a_1 = b_1 a_0 + b_0
    a_1$. Тогда $b_1$ делится на $p$. Аналогично для всех остальных $b_i$. Тогда $p | b_m$, следовательно $a_n = b_m
    c_k$ также делится на $p$. Противоречие.
\end{proof}

\begin{statement}
    Многочлен $\Phi_p(x) = x^{p - 1} + x^{p - 2} + \cdots + x + 1$, где $p$ -- простое число, неприводим.
\end{statement}

\begin{proof}
    $\Phi_p(x) = \frac{x^p - 1}{x - 1}$. Сделаем замену $t = x - 1$. Тогда $\Phi_p = \frac{(t + 1)^p - 1}{t} = \sum_{i =
    1}^p \binom{p}{i} t^{i - 1} = t^{p - 1} + \sum_{i = 2}^{p - 1} \binom{p}{i} t^{i - 1} + p$ -- неприводим по критерию
    Эйзенштейна.
\end{proof}

\begin{statement}
    Пусть $F$ -- факториальное кольцо. Тогда неразложимые многочлены $F[x]$ степени 0 -- это в точности простые элементы
    $F$.
\end{statement}

\begin{proof}
    Пусть $p$ -- неразложимый в $F[x]$. Тогда $\forall a, b, p = ab:$ без ограничения общности $a \in F[x]^*$. Но $a, b$
    -- тоже многочлены степени 0, то есть константы. Но тогда $a \in F^*$. Значит, $p$ -- неразложимый, а следовательно,
    простой в $F$.

    Пусть наоборот, $p$ -- простой элемент $F$. Пусть он разложим над $F[x]$, то есть $\exists f, g \not \in F[x]^*: fg
    = p$. Тогда степень $f$ и $g$ -- ноль. Тогда $p = fg$ над $F$. То есть $p$ -- не простой.
\end{proof}

\begin{statement}
    Для произвольных многочленов $f, g \in F[x]$ над факториальным кольцом $F$ выполнено равенство $c(fg) = c(f)c(g)$. 
\end{statement}

\begin{proof}
    Пусть $fg = c(f) \hat f c(g) \hat g$, где $\hat f, \hat g$ -- примитивные. Докажем, что $\hat f \hat g$ -- тоже
    примитивный. Пусть $\hat f \hat g = a^n x^n + \cdots + a_0, \hat f = b_m x^m + \cdots + b_0, \hat g = c_l x^l +
    \cdots + c_0$. Тогда $a_k = \sum_{i = 0}^k b_i c_{l - i}$. Предположим, что $\hat f \hat g$ -- не примитивный. Тогда
    $\exists p: p | c(\hat f \hat g)$, $p$ -- простое. 
\end{proof}

\end{document}
