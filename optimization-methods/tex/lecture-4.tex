\documentclass[document.tex]{subfiles}

\begin{document}
\section{Выпуклые оптимизации}
\begin{definition}
    $M \subset \mathbb{R}^n$ -- выпукло, если $\forall x, y \in M: \forall \alpha \in (0, 1): \alpha x + (1 - \alpha) y
    \in M$
\end{definition}

\begin{example}
    ~\begin{enumerate}
        \item Шар -- выпуклое множество
        \item Плоскость
        \item Симплекс
    \end{enumerate}
\end{example}

\begin{statement}
    Если $M$ -- выпукло, а $x_1, \cdots, x_m \in M$, то $\alpha_1 x_1 + \cdots + \alpha_m x_m \in M$, если $\alpha_i
    \geq 0, \sum_{i = 1}^n \alpha_i = 1$
\end{statement}

\begin{proof}
    Доказываем индукцией по $m$.

    База: $m = 2$.

    Переход: $m > 2$.
    Рассмотрим $y_1 =\frac{\alpha_1}{1 - \alpha_m} x_1 + \cdots + \frac{\alpha_{m - 1}}{1 - \alpha_m} x_{m - 1}, y_2 = x_m$
    $y_1 \in M$ по предположению индукции, $y_m \in M$. Тогда $(1 - \alpha_m)y_1 + \alpha_m y_2 \in M$.
\end{proof}

\begin{statement}
    Пересечение выпуклых множеств -- выпукло
\end{statement}

\begin{proof}
    Пусть $\beta \in (0, 1)$. Пусть $\{A_i : i \in I\}$ -- набор выпуклых множеств. Пусть $x, y \in \cap A_i$. Тогда
    $\forall i \in I: x, y \in A_i$. Поскольку $A_i$ -- выпуклые, то $\forall i \in I: \beta x + (1 - \beta) y \in A_i$.
    Значит, $\beta x + (1 - \beta) y \in \cap A_i$
\end{proof}

\begin{definition}
    Выпуклая оболочка множества $A$ -- $\langle A \rangle$ -- наименьшее по включению выпуклое множество, содержащее $A$.
\end{definition}

\begin{remark}
    Заметим, что $\langle A \rangle$ -- это пересечение всех выпуклых множеств, содержащих множество $A$
\end{remark}

\begin{statement}
    $\langle A \rangle = \{\alpha_1 x_1 + \cdots + \alpha_m x_m : \sum \alpha_i = 1, \alpha_i \geq 0, x_1, \cdots, x_m
    \in A\} = B$
\end{statement}

\begin{proof}
    Заметим, что $B$ выпукло. Значит, $\langle A \rangle \subset B$. Но кроме того, $\langle A \rangle \supset B$.
\end{proof}

\begin{statement}
    Если $M$ -- выпуклое, то $\overline M$ -- выпукло, $Int M$ -- выпуклое.
\end{statement}

\begin{proof}
    Пусть $x, y \in \overline M$. Кроме того, $x, y \in \overline M \setminus M$. Пусть $x_n \rightarrow x$, $y_n
    \rightarrow y$. Тогда $\alpha x_n + (1 - \alpha) y_n \in M$. Но тогда и $\alpha x + (1 - \alpha)y \in M$.

    Пусть $x, y \in Int M$. $\langle U_{\varepsilon}(x) \cup U_{\varepsilon}(y) \rangle \supset$ отрезок $[x, y]$.
    Тогда любая точка отрезка -- внутренняя.
\end{proof}

\begin{theorem}[Теорема об отделимости]
    Для замкнутого выпуклого $M$ и точки $x \not \in M$ $\exists z : \forall y \in M : (y, z) > (x, z)$
\end{theorem}

\begin{proof}
    Пусть $y_0 \in M, |y_0 - x| = \min_{y \in M} |y - x|$. То есть точка $y_0$ -- ближайшая из $M$ к точке $x$. Тогда
    $\forall y \in M : |y_0 - x| \leq |y - x|$. Возьмем $t  = \alpha y + (1 - \alpha) y_0$. Тогда $|y_0 - x|^2 \leq
    |\alpha y + (1 - \alpha)y_0 - x|^2 = (\alpha y + (1 - \alpha) y_0 - x, \alpha y + (1 - \alpha) y_0 - x) = \alpha^2
    |y - y_0|^2 + |y_0 - x|^2 + 2 \alpha (y - y_0, y_0 - x)$
    Тогда $\alpha^2 |y - y_0|^2 + 2 \alpha (y - y_0, y_0 - x) \geq 0$. Положим $z = y_0 - x$. Сократим на $\alpha$
    $\alpha |y - y_0 + 2 (y - y_0, y_0 - x) \geq 0$ и положим $\alpha = 0$. Так как множество замкнуто, то так сделать
    можно. $(y - y_0, y_0 - x) \geq 0$. $(y - y_0, z) \geq 0$. Получили, что $(y, z) \geq (y_0, z)$. Докажем, что $(y_0,
    z) > (x, z)$. $(y_0, z) - (x, z) = (y_0 - x, z) = (z, z) > 0$. Теорема доказана.
\end{proof}
\begin{theorem}
    Если $M_1, M_2$ -- два замкнутых выпуклых не пересекающихся множества. Тогда их можно отделить друг от друга
    гиперплоскостью. То есть $\exists z: \forall y_1 \in M_1, y_2 \in M_2: (y_1, z) < (y_2, z)$
\end{theorem}

\begin{proof}
    Пусть $y_1^*$ -- ближайшая к $M_2$ точка из $M_1$, $y_2^*$ -- ближайшая к $M_1$ точка из $M_2$. Найдется $z: (y_1^*,
    z) < (y_2, z)$

    Рассмотрим $M_1 - M_2$ -- разность Минковского. $0 \not \in M_1 - M_2$. Тогда $\exists z: (0, z) \leq (y_1 - y_2,
    z)$. Тогда $(y_1, z) < (y_2, z)$
\end{proof}

\end{document}
