\documentclass[document.tex]{subfiles}

\begin{document}
\section{}
\subsection{Вероятность на статистической модели}

\begin{definition}
	Множество всех возможных значений наблюдения называется выборочным пространством и обозначается $\mathfrak{X}$
\end{definition}

\begin{definition}
	Наблюдение -- это результат случайного выбора элемента из выборочного пространства
\end{definition}

Наша цель -- по наблюдению $\mathfrak{X}$ сделать выводы о его распределении $D$.

\begin{definition}
	Выборка размера $n$ из неизвестного распределения -- это набор $X_1, \dots, X_n$ -- независимо распределенных случайных величин.
\end{definition}

\begin{definition}
	Пусть $X = (X_1, \dots, X_n)$ выорка на некотором вероятностном пространстве $(\Omega, \mathcal{F}, P)$ с неизвестным распределением $D$
	Для каждого множества $B$ найдем $P_n^* = \frac{\mu_n(B)}{b}$, где $\mu_n(B)$ -- это количество элементов из $X_1, \dots, X_n$, которые попали в $B$.
	Такая мера называется эмпирическим распределением.
\end{definition}

\begin{statement}
	Пусть $P$ -- неизвестное распределение $X_i$. Тогда $\forall B \in \mathcal{F}: P_n^*(B) \cae P(B)$.
\end{statement}

\begin{proof}
	$$P_n^*(B) = \frac{1}{n} \sum_{i = 1}^n I\{X_i \in B\} \cae P(B)$$ по закону больших чисел.
\end{proof}

\begin{definition}
	Пусть $X = (X_1, \dots, X_n$ -- выборка. 
	$$F_n(x) = \frac{1}{n} \sum_{i = 1}^n I\{X_i \leq x\}$$
	называется эмпирической функцией распределения (она является функцией распределения для эмпирического распределения.
\end{definition}

\begin{theorem}
	$$\sup_x |F_n(x) - F(x)| \cae 0$$
\end{theorem}

\begin{proof}
	Зафиксируем элементарный исход $\omega$. Посмотрим на функцию распределения $F_n(x)$. Она является непрерывной справа. $F(x)$ тоже непрырывна справа, потому что она является функцией распределения. Их разность тоже непрерывна справа. $\sup_x |F_n(x) - F(x)| = \sup_{x \in \mathbb{Q}} |F_n(x) - F(x)|$ -- это супремум счетного числа случайных величин. Поэтому эта величина также является случайной. Пусть $N \in \mathbb{N}$ -- достаточно большое. Введем $x_{k, N} = \min \{x : F(x)  \geq \frac{k}{N}\}$. Полагаем также $x_{0, N} = -\infty, x_{N, N} = +\infty$. Оценим $F_n(x) - F(x)$:
	$$\exists k : F_n(x) - F(x) \leq F_n(x_{k+1, N} - 0) - F(x_{k, N}) =$$
	$$= F_n(x_{k+1, N} - 0) - F(x_{k+1, N} - 0) + F(x_{k+1, N} - 0) - F(x_{k, N}) \leq F_n(x_{k+1, N} - 0) - F(x_{k+1, N} - 0) + \frac{1}{n}$$
	Получаем, что
	$$\sup_{x \in \mathbb{Q}} |F_n(x) - F(x)| \leq \max \{ |F_n(x_{k, N}) - F(x_{k, N} - 0)|, F_n(x_{k+1, N} - 0) - F(x_{k+1, N} - 0) + \frac{1}{n}\} \leq \cdots$$
	
\end{proof}

\end{document}

