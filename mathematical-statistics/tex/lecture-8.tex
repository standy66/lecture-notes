\documentclass[document.tex]{subfiles}

\begin{document}
\section{Условное распределение}
\begin{definition}
    Пусть $A \in \mathcal{F}$, $\mathcal{G} \subset \mathcal{F}$ -- под-сигма-алгебра. Определим $P(A|G) := E(I_A|G)$
\end{definition}

\begin{definition}
    Пусть $\eta$ -- случайная величина, $\mathcal{F}_{\eta}$ -- сигма-алгебра, порожденная случайной величиной $\eta$.
    Тогда $E(\xi|\eta) := E(\xi|\mathcal{F}_{\eta})$
\end{definition}

\begin{definition}
    Пусть $\xi, \eta$ -- случайные величины. Тогда $E(\xi | \eta = y)$ -- это такая борелевская $\varphi(y)$, что 
    $\forall B \in \mathcal{B}(\mathbb{R}): E(\xi I \{\eta \in B\}) = \int_B \varphi(y) P_{\eta}(dy)$, где $P_{\eta}$ --
    это распределение случайной величины $\eta$
\end{definition}

\begin{lemma}
    Пусть $E|\xi| < \infty$, тогда $E(\xi|\eta = y)$ существует и единственно почти наврное.
\end{lemma}

\begin{proof}
    Следует из теоремы Радона-Никодима.
\end{proof}

\begin{statement}
    $E(\xi | \eta = y) = \varphi(y) \Leftrightarrow E(\xi | \eta) = \varphi(\eta)$
\end{statement}

\begin{proof}
    Докажем сначала слева направо. Ясно, что $\varphi(\eta)$ является $\eta$-измеримой. Проверим интегральное свойство.
    Мы значем, что $E(\xi I\{\eta\in B\}) = \int_B \varphi(y) P_{\eta}(dy)$. Пусть $A \in \mathcal{F}_{\eta}$. Тогда $A
    = \{\eta \in B\}$ для некоторого $B \in \mathcal{B}(\mathbb{R})$. Тогда $E\xi I_A = E\xi I\{\eta \in B\} = \int_B
    \varphi(y) P_{\eta}(dy) = \int_{\mathbb{R}} \varphi(y) I_{B} P_{\eta}(dy) = E \varphi(\eta) I\{\eta \in B\}$.
    Последнее равенство выполнено в силу теоремы о замене переменной под интегралом Лебега.
    В обратную сторону утверждение доказывается аналогично (только теперь нам извество, что $\varphi(\eta)$ -- это
    условное матожидание).
\end{proof}

\begin{corollary}
    $\xi$ является $\eta$ измеримой $\Leftrightarrow \xi = \varphi(\eta)$ для некоторой борелевской $\varphi$.
\end{corollary}

\begin{definition}
    Условным распределением случайной величины $\xi$ при условии $\eta = y$ называется $P(\xi \in B | \eta = y)$
    рассматриваемая как функция от $B$.
\end{definition}

\begin{statement}
    При фиксированном $y$ величина $P(\xi \in B | \eta = y)$ является вероятностной мерой $P_{\eta}$ почти наверное на $(\mathbb{R},
    \mathcal{B}(\mathbb{R}))$
\end{statement}

\begin{definition}
    Если условное распределение предствимо в виде $P(\xi \in B | \eta = y) = \int_B f_{\xi | \eta}(x | y) dx$, где
    функция $f_{\xi | \eta}(x | y)$ -- это обычная неотрицательная функция двух переменных, то говорят, что
    $f_{\xi | \eta}$ является плотностью условного распределения.
\end{definition}

\begin{theorem}
    Пусть $g(x)$ -- борелевская функция, такая что $E|g(\xi)| < \infty$. Тогда при условии существования условной
    плотности $f_{\xi | \eta}$ можно записать следующее:
    \[
        E(g(\xi) | \eta = y) = \int_{\mathbb{R}} g(x) f_{\xi | \eta}(x | y) dx
    \]
\end{theorem}

\begin{proof}
    Coming soon
\end{proof}

\begin{proof}
    Нетрудно проверить, что $\int_A \varphi(x, y) dx$ является условным распределением $P(\xi \in A | \eta = y)$. Тогда
    $\forall B \in \mathcal{B}(\mathbb{R}): E(I\{f \in A\} I\{\eta \in B\}) = P(\xi \in A, \eta \in B) = \int_{A \times
    B} f_{\xi, \eta}(x, y) dx dy  = \int_B \left( \int_A f_{\xi, \eta}(x, y) f_{\eta}^{-1}(y) dx \right) f_{\eta}(y)
    dy = \int_B \left( \int_A \varphi(x, y) dx \right) P_{\eta}(dy) = \int_B P(\xi \in A | \eta = y) P_{\eta}(dy)$.
    Значит, $\varphi(x, y)$ -- действительно условная плотность.
\end{proof}

Как искать условное математическое ожидание:

\begin{enumerate}
    \item Находим совметсную плотность $f_{(\xi, \eta)} (x, y)$.
    \item Находим плотность условия $f_{\eta}(y) = \int_{\mathbb{R}}f_{(\xi, \eta)}(x, y) dx$
    \item Находим условную плотность $f_{\xi | \eta}(x|y) = \frac{f_{(\xi, \eta)}(x, y)}{f_{\eta}(y)}$
    \item $\varphi(y) = E(g(\xi) | \eta = y) = \int_{\mathbb{R}} g(x) f_{\xi | \eta}(x|y) dx$
    \item $E(g(\xi) | \eta) = \varphi(\eta)$
\end{enumerate}

\begin{statement}
    В дискретном случае $E(g(\xi) | \eta = y) = \sum_{x \in Im \xi} g(x) P (\xi = x | \eta = y)$
\end{statement}

\begin{proof}
    Упражнение.
\end{proof}

\end{document}

