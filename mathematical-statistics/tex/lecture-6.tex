\documentclass[document.tex]{subfiles}

\begin{document}
\begin{corollary}
    В условиях предыдущей теоремы, если $\forall x:$ решение ОМП единственно, то ОМП асимптотически нормальная с наилучшей асимптотической дисперсией.
\end{corollary}

\begin{theorem}[Бахадур]
    В условиях предыдущей теоремы, $\hat \theta_n$ -- асимптотически нормальная оценка, $\sqrt{n}(\hat \theta_n -
    \theta_0) \cd N(0, \sigma(\theta)^2)$
\end{theorem}

\begin{corollary}
    В условиях предыдущей теоремы, и предыдущего следствия, ОМП является наилучшей оценкой в асимптотическом подходе в
    классе оценок с непрерывной асимптотической диспресией.
\end{corollary}

\begin{definition}
    Оценка $\hat \theta_n$ называется асимптотически эффективной, если $\sqrt{n}(\hat \theta_n - \theta) \cd N(0,
    \frac{1}{i(\theta)}$
\end{definition}

\begin{theorem}[Эффективность ОМП]
    Пусть выполнены условия регулярности, $\hat \theta$ -- эффективная оценка $\theta$. Тогда $\hat \theta$ -- ОМП.
\end{theorem}

\begin{proof}
    Вспомним, что оценка эффективная, когда $\hat \theta(X) - \theta = \frac{1}{I(\theta)}\frac{\partial}{\partial
    \theta}f_{\theta}(X)$ Если $\theta > \hat \theta$, то $f_{\theta}$ убывает, если $\theta < \hat \theta$, то
    $f_{\theta}$ возрастает. Значит, $\theta = \hat \theta$ -- это ОМП.
\end{proof}

\subsection{Условное матожидание}
\begin{definition}
    Пусть $(\Omega, \mathcal{F}, P)$ -- вероятностной пространство. Пусть $\xi$ -- случайная величина, а
    $\mathcal{G}$ -- это сигма алгебра. $\zeta$ называется условным матожиданием $\xi$ по алгебре $\mathcal{G}$
    (обозначается $E(\xi | \mathcal{G})$), если:
    \begin{enumerate}
        \item $\zeta$ -- является $\mathcal{G}$-измеримой случайной величиной ($\forall B \in
            \mathcal{B}(\mathbb{R}): \zeta^{-1}(B) \in \mathcal{G}$)
        \item $\forall G \in \mathcal{G}: E \xi I_G = E \zeta I_G$
    \end{enumerate}
\end{definition}

\begin{definition}
    $\nu$ -- называется зарядом (или мерой со знаком), если:
    \begin{enumerate}
        \item $\nu: \mathcal{F} \mapsto \mathbb{R}$
        \item $\forall A_1, A_2, \cdots \in \mathcal{F}$ -- непересекающихся, $\nu(\cup A_i) = \sum \nu(A_i)$
    \end{enumerate}
\end{definition}

\begin{theorem}[Радон, Никодим]
    Пусть $\nu$ -- заряд на $\mathcal{G}$. Тогда $\exists \zeta$ -- случайная величина, такая что $\forall G \in
    \mathcal{G}$: $\int_{G}^{}\zeta P(d \omega) = \nu(A)$, причем $\zeta$ является $\mathcal{G}$-измеримой.
\end{theorem}

\begin{theorem}[существование условного матожидания]
    Пусть $\xi: E\xi < \infty$. Тогда $\exists \zeta = E(\xi | \mathcal{G})$
\end{theorem}

\begin{proof}
    Возьмем в качестве $\nu(A) = E \xi I_{A}$. Нетрудно убедится, что из теоремы Лебега о мажорируемой сходимости
    следует, что это действительно заряд. 
    $|\xi \sum_{i = 1}^k I_{A_i}| < \xi$. Но $\xi \sum_{i = 1}^k I_{A_i} \cae \xi \sum_{i = 1}^{\infty}$. По теореме
    Радона Никодима $\exists \zeta: E \zeta I_A = \nu(A)$
\end{proof}

\begin{example}
    Берём $\mathcal{G}$ -- разбиение на $G_1, G_2, \cdots$. Тогда $E(\xi|\mathcal{G}) = \sum_{i = 1}^{\infty} c_i
    I_{G_i}$ из первого свойства, а из второго свойства $c_i = \frac{E\xi I_{G_i}}{P(G_i)}$
\end{example}

Свойства:
\begin{enumerate}
    \item Условное матожидание единственно почти наверное. Действительно, пусть $\zeta_1, \zeta_2$ -- два матожидания.
        $0 \leq E(\zeta_1 - \zeta_2)I_A \leq 0$. Значит, $\zeta_1 \leq \zeta_2$ почти наверное, $\zeta_1 \geq \zeta_2$
        почти наверное.
    \item Если $\xi$ -- $\mathcal{G}$-измерима, то $E(\xi|\mathcal{G}) = \xi$
    \item $E(E(\xi|\mathcal{G})) = E\xi$
    \item Если $\xi$ не зависит от $\mathcal{G}$, то $E(\xi|\mathcal{G}) = E\xi$. Действительно, рассмотрим $\zeta =
        E\xi$. Оно удовлетворяет обоим свойствам условного матожидания.
    \item $E(\alpha \xi_1 + \beta \xi_2 | \mathcal{G}) = \alpha E(\xi_1 | \mathcal{G}) + \beta E(\xi_2 |
        \mathcal{G})$
\end{enumerate}

\end{document}

