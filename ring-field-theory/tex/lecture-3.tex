\documentclass[document.tex]{subfiles}

\begin{document}
\begin{statement}
	Целые Гауссовы числа и числа Эйзенштейна -- это евклидовы кольца.
\end{statement}

\begin{theorem}
	Евклидовы кольца факториальны:
	(1) доказываем, что разложение существует
	(2) неразложимый элемент простой
\end{theorem}

\begin{statement}
	Если $K$ -- евклидово кольцо, то выполнено свойство (1).
\end{statement}

\begin{proof}
	Пусть это не вполнено. Расрим тай необрай элемент $x \neq 0$, что его норма минимальна среди всех элементов, для которых свойство (1) не выполнено. Пусть $x = x_1 x_2$. Если для любого такого разложения $x_1 | x_2 \in K^*$, тогда $x$ неразложим. Проотиворечие. Пусть $x = x_1 x_2$, т.ч. $x_1, x_2 \not \in K^*$. $N(x) = N(x_1 x_2) = N(x_2) \Leftrightarrow x_2 \in K^*$. Тогда $N(x_1) < N(x), N(x_2) < N(x)$. Раскладываем $x_1, x_2$ на простые. Противоречие.
\end{proof}

\begin{definition}
	$a, b \in K \setminus \{0\}$. $(a, b) = x : a | x, b | x, N(x)$ максимальна
\end{definition}<++>

\end{document}
