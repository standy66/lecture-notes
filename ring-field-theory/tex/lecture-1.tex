\documentclass[document.tex]{subfiles}

\begin{document}

Курс состоит из 3 частей:
\begin{enumerate}
\item Теория делимости. Обобщение ОТА (основная теорема арифметики).
\item Расширения полей. Основная теорема алгебры. Конечные поля. Коды БЧХ.
\item Как из $\mathbb{Q}$ перейти в $\mathbb{R}$. $\mathbb{Q}_p$.
\end{enumerate}

\section{Базовые определения.}

\begin{definition}
Кольцо - это тройка $(K, +, \cdot)$. Причем $(K, +)$ -- абелева группа, и:
\begin{align*}
	\forall a, b, c \in K: &(a + b) \cdot c = a \cdot c + b \cdot c &	\\
			     		&c \cdot (a + b) = c \cdot a + c \cdot b
\end{align*}
\end{definition}

\begin{definition}
Говорят, что кольцо $K$ обладает 

\begin{enumerate}
	\item Ассоциативностью, если выполнено $\forall a, b, c \in K: (ab)c = a(bc)$
	\item Нейтральным элементом, если выполнено $\exists 1 \in K: \forall a \in K: a \cdot 1 = 1 \cdot a = a$
	\item Коммутативностью, если выполнено $\forall a, b \in K: ab = ba$
\end{enumerate}
\end{definition}

\begin{definition}
Коммутативное кольцо -- это такое кольцо, что для умножения выполнена коммутативность и (внезапно) ассоциативность и существование нейтрального элемента.
\end{definition}

\begin{remark}
В дальнейшем буквой $K$ будем обозначать коммутативное кольцо (т.е. коммутативное с единицей и ассоциативностью).
\end{remark}

\begin{example}
~\begin{enumerate}
\item $\mathbb{Z}$ является коммутативным кольцом с единицей и ассоциативностью
\item $\{0\}$ -- тривиальное кольцо
\item $2\mathbb{Z}$ -- кольцо без единицы, но ассоциативное и коммутативное.
\item $\mathbb{R}^{n \times n}$ -- ассоциативная кольцо с единицей, но не коммутативное
\end{enumerate}
\begin{example}
Более интересный пример:
Множество матриц со сложением и операцией $[\cdot, \cdot]$:
$[A, B] = AB - BA$.
Ассоциативность не выполнена. Но выполнено:
\begin{enumerate}
\item $[[A, B], C] + [[B, C], A] + [[C, A], B] = 0$
\item $[A, B] = -[B, A]$
\end{enumerate}
\end{example}
\end{example}

\begin{definition}
Пусть $K$ -- коммутативное кольцо. Тогда $a \neq 0$ называется делителем нуля, если:
$\exists b \neq 0 : ab = 0$.
\end{definition}

\begin{definition}
Коммутативное кольцо без делителей нуля называется областью целостности.
\end{definition}

\begin{statement}
$a \cdot 0 = 0$
\end{statement}

\begin{proof}
	$a \cdot 0 = a \cdot (0 + 0) = a \cdot 0 + a \cdot 0$. Прибавляя обратный по сложению для $a \cdot 0$, получаем, что $a \cdot 0 = 0$
\end{proof}

\begin{definition}
$F$ -- поле, если:
\begin{enumerate}
\item $F$ -- ассоциативное коммутативное кольцо с единицей
\item $1 \neq 0$
\item Любой элемент обратим относительно умножения.
\end{enumerate}
\end{definition}

\begin{statement}
В поле нет делителей нуля.
\end{statement}
\begin{proof}
Пусть $a$ -- делитель нуля, т.е. $\exists b \neq 0 : ab = 0$. Но у $a$ есть обратный элемент относительно умножения $a^{-1}$. Умножив слева на $a^{-1}$, придем к противоречию.
\end{proof}

\begin{definition}
Гауссовы числа ($\mathbb{Z}[i]$) -- это комплексные числа с целой мнимой и действительной частью.
\end{definition}

\begin{statement}
Гауссовы числа -- это область целостности
\end{statement}
\begin{proof}
Замкнутость относительно операций проверяется тривиальным образом. Коммутативность, дистрибутивность и ассоциативность следует из соответствующих свойств для $\mathbb{C}$. $0 + 0i$ -- нейтральный элемент относительно сложения, а $1 + 0i$ -- нейтральный элемент относительно умножения, проверяется тривиальным образом. А делителей нуля в гауссовых числах нет, поскольку что их нет в комплексных числах ($\mathbb{C}$ -- это поле).
\end{proof}

\begin{definition}
В области целостности $K$ говорят, что $a|b$ ($a$ делит $b$), если $\exists c \neq 0 : ac = b$.
\end{definition}

\begin{statement}
Свойства делимости: для любых ненулевых $a, b, c$:
\begin{enumerate}
\item $a|b, b|c \Leftarrow a|c$
\item Если $b + c \neq 0$, то $a|b, a|c \Leftarrow a|(b+c)$
\item $a|1 \Leftrightarrow a \text{ -- обратимый элемент }$
\end{enumerate}
\end{statement}

\begin{proof}
	~\begin{enumerate}
		\item $a|b, b|c$ эквивалентно $\exists q_1, q_2 \neq 0 : aq_1 = b, bq_2 = c$. Следовательно, $a q_1 q_2 = c$, причем $q_1 q_2$ -- не ноль.
		\item $a|b, a|c$ эквивалентно $\exists q_1, q_2 \neq 0 : aq_1 = b, aq_2 = c$. Следовательно, $a q_1 + a q_2 = a (q_1 + q_2) = b + c$. Причем $q_1 + q_2$ -- не ноль, в противном случае $b+c = 0$
		\item $a|1$ эквивалентно $\exists b \neq 0: ab = 1$ эквивалентно тому, что $a$ -- обратимый
	\end{enumerate}
\end{proof}

\begin{remark}
В случае, когда $a|1$, любой элемент поля делится на $a$:

$x = 1 \cdot x = a \cdot a^{-1} \cdot x$
\end{remark}

\begin{definition}
$K^*$ (множество обратимых элементов ассоциативного кольцо с единицей K) называют мультипликативной группой кольца.
\end{definition}

\begin{proof}
	Поскольку кольцо было ассоциативное и с единицей, то ассоциативность операции группы и существование единицы выполнены.
	Поскольку в $K^*$ только обратимые элементы, то операция в группе также обратимы. Отсалось проверить замкнутость, то есть $\forall a, b \in K^*: ab \in K^*$
	Это действительно так, обратный к $ab$ -- это $b^{-1}a^{-1}$ 
\end{proof}

\begin{definition}
Будем называть два элемента $a$ и $b$ ассоциированными (обозначение: $a \sim b$), если $a = rb, r \in K^*$.
\end{definition}

\begin{exercise}
$~$ -- это отношение эквивалентности.
\end{exercise}

\begin{proof}
	~\begin{enumerate}
		\item $a = rb \rightarrow r^{-1}a = b$
		\item $a = rb, b = sc \rightarrow a = rsc$
		\item $a = 1 \cdot a$
	\end{enumerate}
\end{proof}
\end{document}
