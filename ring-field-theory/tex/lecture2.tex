\documentclass[document.tex]{subfiles}

\begin{document}
\section{Факториальные кольца.}
\begin{definition}
	Область целостности $K$ называется факториальным кольцом, если:
	\begin{enumerate}
		\item $\forall x \neq 0: \exists u \in K^*, p_1, \ldots, p_k \text{ -- неприводимые} : x = up_1p_2 \ldots p_k$
		\item Если существует два разложения, то они равны по подулю перестановки и ассоциируемости
	\end{enumerate}
\end{definition}

\begin{remark}
	Чтобы доказать, что область целостности является факториальным, нужно выполнить 3 шага:
	\begin{enumerate}
		\item $\exists$ разложение
		\item Доказываем, что каждый неразложимый элемент -- простой
		\item Доказываем единственность разложения
	\end{enumerate}
\end{remark}

\begin{statement}
	Простой элемент неразложим.
\end{statement}

\begin{proof}
	Пусть $x = ab$ -- простой. Тогда $a|x, b|x$. Кроме того $x | ab$. Если $x|a$, то $x \approx a$. А значит, $b \in K^*$. Если же $x \approx b$, то проводим аналогичное доказательство. 
\end{proof}

\begin{remark}
	Обратное верно не всегда.
\end{remark}

\begin{statement}
	Если для кольца мы уже доказали п.1 и п.2, то единственность разложения будет из этого следовать.
\end{statement}

\begin{proof}
	Мы хотим доказать единственность. Пусть $x = up_1 \ldots p_k, x = vq_1 \ldots q_l$, где $u, v \in K^*$. Возьмем какое-нибудь $p_i$, если $\exists q_j: q_j \approx p_i$, то их сократим, и так далее, пока можем. Получили, что какое-нибудь $p_i | w q_{j_1} q_{j_2} \ldots q_{j_s}$. Поскольку $p_i$ простое, то получим, что $p_i | q_j$. Тогда $p_i u = q_j$, но така как $q_j$ неразложим, получаем, что $u \in K^*$. А значит, $p_i \approx q_j$. Противоречие
\end{proof}

\begin{definition}
	Область целостности K называется Евклидовым кольцом, если: $\exists ||x|| : K \setminus \{0\} \mapsto \mathbb{N}_0$ -- норма, для которой выполнено:
	\begin{enumerate}
		\item $\forall a, b \neq 0: ||ab|| \geq ||a||$
		\item $\forall a, b \neq 0: \exists q, r \in K : a = bq+r \Rightarrow (r = 0 \vee ||r|| < ||b||)$
	\end{enumerate}
\end{definition}

\begin{statement}
	Свойство 1 лишнее.
\end{statement}

\begin{proof}
	Положим
	$$N(a) = \min_{b \neq 0, b \in K} ||ab||$$
	Заметим, что свойство 1 выполнено. Докажем, что свойство 2 выполнено: пусть $0 \neq a, b \in K$. $N(b) = ||bc||$. Разделим $a$ на $bc$ с остатком. $a = q(bc) + r$. Если $r \neq 0$, то $N(r) \leq |r| < ||bc|| = N(b)$
	
\end{proof}

\begin{example}
	~\begin{enumerate}
		\item $\mathbb{Z}$
		\item $K[x_1, \ldots, x_n]$, где $K$ -- поле, $||P|| = deg P$
		\item $\mathbb{Z}[i]$
	\end{enumerate}
\end{example}

\end{document}
