\documentclass[document.tex]{subfiles}

\begin{document}
\begin{lemma}[Усиленная лемма Гронуолла]
    Пусть на интервале $I \subset R$. Некоторая функция $\varphi(x) \geq 0$ непрерывна и удовлетворяет следующему
    свойству:
    \[
        \exists A \geq 0, B > 0, C \geq 0:
        \forall x \in I: \varphi(x) \leq A + B \| \int_{x_0}^{x} \varphi(\tau) d\tau \| + C \|x - x_0\|.
    \]
    Тогда
    \[
        \forall x \in I : \varphi(x) \leq A e^{B \|x - x_0 \|} + \frac{C}{B} (e^{B\|x - x_0\|} - 1)
    \]
\end{lemma}
\begin{proof}
    Пусть $x > x_0$. Обозначим $\Phi(x) = \int_{x_0}^{x}\varphi(\tau) d\tau$. Тогда $\Phi(x) \geq 0$, $\Phi'(x) \geq
    0$, $\Phi(x_0) = 0$. Из условия следует, что 
    \[
        0 \leq \Phi'(x) \leq A + B \Phi(x) + C(x - x_0)
    \]
    Умножим обе части неравенства на $e^{-B(x - x_0)}$.
    \[
        0 \leq \Phi'(x) e^{-B(x - x_0)} \leq  (A + B \Phi(x) + C(x - x_0)) e^{-B(x - x_0)}
    \]
    Переносим $B \Phi(x) e^{-B(x - x_0}$ в левую часть и замечаем, что это производная от $\Phi(x) e^{-B(x - x_0)}$
    \[
        \frac{d}{dx} \Phi(x) e^{-B(x - x_0)} \leq (A + C(x - x_0)) e^{-B(x - x_0)}
    \]
    Проинтегрируем это неравенство. В качестве левой части получаем:
    \[
        \int_{x_0}^{x} \frac{d}{d\tau} \Phi(\tau) e^{-B(\tau - x_0)} d \tau = \Phi(x) e^{-B(x - x_0)} - \Phi(x_0)
        e^{-B(x - x_0)} = \Phi(x) e^{-B(x - x_0)}
    \]
    Правую часть интегрируем по частям. 
    \begin{multline*}
        \int_{x_0}^{x} (A + C(\tau - x_0)) e^{-B(\tau - x_0)} = -\frac{A}{B} ( \int_{x_0}^{x} (A + C(\tau - x_0))
            \\
        e^{-B(\tau - x_0)} ) = -\frac{A}{B} (  ( (A + C(x - x_0)) e^{-B(x - x_0)} - A) + \\
        \frac{C}{B} (e^{-B(x - x_0)} - 1) )
    \end{multline*}

    Получили ограничение вида $\Phi(x) \leq \cdots$. Подставляя его в неравенство из условия, получаем то, что нужно
\end{proof}

\begin{theorem}[о продолжении решения уравнения на интервал]
    Пусть вектор функция $f(x, y)$ удовлетворяет условиям теоремы о существовании и единственности на множестве $G = \{(x,
        y): x \in (\alpha, \beta), y \in \mathbb{R}^{n}\}$. При этом существуют функции $a(x)$, $b(x)$ непрерывные на
        $(\alpha, \beta)$, такие что $\|f(x, y)\| \leq a(x)\|y\| + b(x)$. Тогда каждое решение задачи Коши 
        \[
            \begin{cases}
                y' = f(x, y), \\
                y_0 = y(x_0),
            \end{cases}
        \]
        , где $(x_0, y_0) \in G$ можно продолжить на весь интервал $(\alpha, \beta)$
\end{theorem}

\begin{proof}
    Заметим, что достаточно доказать для задачи Коши при начальной точке $(x_0, y_0) = (x_0, 0)$. Пусть $x_0 \in
    (\alpha, \beta)$. Пусть $[\alpha_1, \beta_1] \subset (\alpha, \beta)$, $x_0 \in [\alpha_1, \beta_1]$ Мы знаем, что
    любое решение задачи Коши можно продолжить на $[\alpha_1, \beta_1]$. Пусть $y = \int_{x_0}^{x}f(\tau, y(\tau))
    d\tau$. Тогда
    \begin{multline*}
        \|y\| = \| \int_{x_0}^{x}f(\tau, y(\tau)) d\tau \| \leq \int_{x_0}^{x} \|f(\tau, y(\tau))\| d\tau \leq \\
        \int_{x_0}^{x}(a(\tau)\|y\| + b(\tau)) d \tau \leq B \|y\| + C(x - x_0)
    \end{multline*}
    По лемме Гронуолла $\|y\| \leq \frac{C}{B} \left[ e^{B(x - x_0)} - 1 \right]$. Обозначим $M = \max_{[\alpha_1,
    \beta_1]} \frac{C}{B} \left[ e^{B(x - x_0)} - 1 \right]$. $G' = \{(x, y) : \alpha_1 \leq \beta_1, \|y\| \leq M +
    1\}$. $\rho(G', G) > 0$. Расширяем наш цилиндр. Получаем решение задачи Коши на объединении, которое стемится к
    $(\alpha, \beta)$
\end{proof}

\subsection{Задача Коши для уравнений I-го порядка, не разрешенных отностительно производной}

\begin{definition}
    \[
        \begin{cases}
            F(x, y, y') = 0, \\
            y(x_0) = y_0, \\
            y'(x_0) = p_0
        \end{cases}
    \] -- задача Коши для уравнения I-го порядка, не разрешенного относительно производной.
\end{definition}

\begin{theorem}
    Пусть в области $G$ $F(x, y, p)$ непрерывно-дифференцируема как функция нескольких переменных, причем
    $\left. \frac{\partial F}{\partial p} \right|_(x_0, y_0, p_0) \neq 0$. Тогда $\exists \delta > 0:$ существует и
    единственно решение задачи Коши на $[x_0 - \delta, x_0 + \delta]$
\end{theorem}

\begin{proof}
    По теореме о неявной функции в некоторой окрестности точки $(x_0, y_0, p_0)$ существует и единственно представление
    $F(x, y, p)$ в виде $p = f(x, y)$, то есть: $p_0 = f(x_0, y_0), F(x, y, f(x, y)) = 0$. Кроме того $f(x, y)$
    дифференцируема в этой окрестности. Используя существование и единственность решения задачи Коши для уравнения,
    разрешенного относительно производной. Получаем, что и для уравнения, не разрешенного относительно производной, тоже
    существует и единственно решение задачи Коши.
\end{proof}

\begin{definition}
    Точки, где задача Коши имеет два или более решений называется точкой локальной неединственности.
\end{definition}

\begin{definition}
    Дискриминантным множеством для данного уравнения первого порядка $F(x, y, p) = 0$, не разрешенного относительно производной называют
    множество точек $\{\frac{\partial F}{\partial p} = 0, F(x, y, p) = 0\}$
\end{definition}

\begin{definition}
    Особым решением уравнения называется такое решение, что в каждой точке $(x, y)$ принадлежащей его интегральной
    кривой, эта интегральная кривая касается интегральной кривой другого решения уравнения и не совпадает с ней в сколь
    угодно малой окрестности точки $(x_0, y_0)$
\end{definition}

\begin{algorithm}
    ~\begin{enumerate}
        \item Отыскиваем дискриминантное множество
        \item Проверяем, является ли это множество решением
        \item Проверка, является ли это решение особым
    \end{enumerate}
\end{algorithm}

\end{document}
