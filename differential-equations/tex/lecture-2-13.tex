\documentclass[document.tex]{subfiles}

\begin{document}
\begin{lemma}[обобщение основной леммы вариационного исчисления]
    Если $f(x)$ -- непрерывна на $[a, b]$ и $\forall \eta(x) \in \mathring{C}^k[a, b]$, и
    \[
        \int_{a}^{b}f(x) \eta(x) dx = 0
    \], то $f(x) \equiv 0$ на $[a, b]$
\end{lemma}

\begin{proof}
    Аналогичное доказательство, функцию $\eta(x)$ только строить сложнее (нужен полином)
\end{proof}

Рассмотрим функционал вида:

\begin{equation}
    \label{kof}
    J(y) = \int_{a}^{b}F[x, y, y', \cdots, y^{(k)}]dx
\end{equation}, причем $F$ -- дважды непрерывно дифференцируема как функция от $x$ и как функция от $y, y', \cdots,
y^{(k)}$. Слабый экстремум определяем аналогично. Если $y$ дает слабый экстремум $J(y)$, то $\forall x \in [a, b]$
$y(x)$ удовлетворяет уравнению Эйлера-Пуассона:
\[
    \frac{\partial F}{\partial y} - \frac{d}{dx} \left( \frac{\partial F}{\partial y'} \right) + \cdots + (-1)^k
    \frac{d^k}{dx^k} \left( \frac{\partial F}{y^{(k)}} \right) = 0
\]


\end{document}
