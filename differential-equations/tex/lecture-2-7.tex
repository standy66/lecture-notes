\documentclass[document.tex]{subfiles}

\begin{document}
\begin{corollary}
    Если в уравнении $y'' + q(x)y = 0$ $q(x) \leq 0$, то любое нетривиальное решение имеет не более 1 нуля.
\end{corollary}

\begin{corollary}
    Пусть $y_1(x), y_2(x)$ -- два линейно-независимых решения (5). Пусть $y_1(x_1) = y_1(x_2) = 0$, где $x_1, x_2$ --
    последовательные нули $y_1(x)$. $x_1 < x_2$. Тогда $\exists ! \hat x \in (x_1, x_2): y_2(\hat x) = 0$
\end{corollary}

\begin{proof}
    Положим $q_1(x) = q_2(x) = q(x)$. Тогда $y_1$ -- решение $y'' + q_1(x)y = 0$, $y_2$ -- решение $y'' + q_2(x)y = 0$.
    Заметим, что $q_1(x) \leq q_2(x)$, $q_2(x) \leq q_1(x)$. Так как $y_1$ и $y_2$ -- ЛНЗ, то $y_2(x_1) \neq 0$,
    $y_2(x_2) \neq 0$. Инача определитель Вронского в этих точках равен нулю. Тогда по теореме Штурма $\exists \hat x:
    y_2(\hat x) = 0$. Заметим также, что других нулей у $y_2$ на этом промежутке быть не может, инача бы нули $x_1, x_2$
    у $y_1$ были бы не последовательными.
\end{proof}

\section{О решении некоторых уравнений второго порядка с помощью аналитических функций}

\begin{theorem}
    Если ряд $\sum_{n = 0}^{\infty} c_n (z - a)^n$, где $c_n, z \in \mathbb{C}$, сходятся хотя бы при одном $z = b \neq
    a$, то $\exists R > 0:$ ряд сходится $\forall z : |z - a| < R$ и расходится $\forall z : |z - a| > R$
    \[
        f(z) = \sum_{i = 1}^{\infty} c_n (z - a)^n
    \] называется аналитической в кругу сходимости $|z - a| < R$. Причем, её можно почленно интегрировать и
    дифференцировать.
\end{theorem}

\begin{theorem}[единственности]
    Если степенные ряды $\sum_{n = 0}^{\infty} a_n(z - a)^n$ и $\sum_{n = 0}^{\infty} b_n(z - a)^n$ сходятся в некотором
    круге сходимости с радиусом $R$, причем $\forall z, |z - a| < R: \sum_{n = 0}^{\infty} a_n (z - a)^n = \sum_{n = 0}^{\infty} b_n
    (z - a)^n$, то коэффициенты $a_n$ и $b_n$ совпадают.
\end{theorem}

\begin{theorem}[об аналитичности линейного ОДУ с аналитическими коэффициентами]
    Пусть $a_1(x), \cdots, a_n(x)$ -- аналитические в каком-то круге $|x - a| < $, тогда решение уравнения $y^{(n)} +
    a_1(x) y^{(n - 1)} + \cdots + a_n(x) y = 0$ является аналитическим в том же самом круге.
\end{theorem}

\begin{definition}[уравнение Эйри]
    $y'' - xy = 0$ называется уравнением Эйри
\end{definition}

\begin{example}
    Попытаемся отыскать решение уравнений Эйри в виде степенного ряда.
    \[
        y(x) = \sum_{n = 0}^{\infty} y_n x^n
    \]
    \begin{multline*}
        y''(x) - xy = \sum_{n = 0}^{\infty} n(n - 1)y_n x^{n - 2} - \sum_{n = 0}^{\infty}y_n x^{n + 1} = \\
        \sum_{n = 0}^{\infty} (n+2)(n+1)y_{n+2} x^{n} - \sum_{n = 1} y_{n - 1} x^{n} = 0
    \end{multline*}
    Если $n = 0$, то $y_{2} = 0$, если $n > 0$, то $y_{n + 2} = \frac{y_{n - 1}}{(n + 2)(n + 1)}$. Автматически
    получаем, что $y_{3k + 2} = 0$. Подбирая начальные условия ($y_0, y_1$) находим два ЛНЗ решения.
    Например, $y_0 = 1, y_1 = 0$. Тогда $y_{3k + 1} = 0$, $y_{3k + 3} = \frac{1}{(2 \cdot 3)(5 \cdot 6 \cdots}$. Причем
    радиус сходимости $R = \infty$ (по формуле Коши-Адамара). Аналогично находим при $y_0 = 0, y_1 = 1$. Получили два
    ЛНЗ аналитических решения с бесконечным радиусом сходимости. Следовательно, любое решение уравнения выражается как
    линейная комбинация этих.
\end{example}

Рассмотрим $y'' + \frac{p_1(x)}{p_2(x)}y' + \frac{q_1(x)}{q_2(x)}y = 0$.

\begin{definition}
    Точки, где хотя бы одно из $p_2(x)$ и $q_2(x)$ обращается в ноль называются особыми точками уравнения. Регулярными
    особой точкой называются такое $\alpha$, что $p_2(x) = (x - \alpha), q_2(x) = (x - \alpha)^2$.
\end{definition}

\begin{definition}
    $x^2y'' + xy + (x^2 - \nu^2)y = 0, \nu = const$ -- уравнение Бесселя.
\end{definition}

\begin{remark}
    Заметим, что 0 явяляется регулярной особой точкой уравнения Бесселя. Попытаемся отыскать решение уравнения Бесселя в
    виде $x^{\alpha}\sum_{n = 0}^{\infty}y_n x^n, \alpha \in \mathbb{R}, y_0 \neq 0$.
    \[
        y' = \sum_{n = 0}^{\infty}(n+\alpha)y_n x^{n + \alpha - 1}
    \]
    \[
        y'' = \sum_{n = 0}^{\infty}(n + \alpha)(n + \alpha - 1) y_n x^{n + \alpha - 2}
    \]
    Подставляем в уравнение, получаем:
    \begin{multline*}
        \sum_{n = 0}^{\infty}(n + \alpha)(n + \alpha - 1) y_n x^{n + \alpha} + \sum_{n = 0}^{\infty}(n + \alpha) y_n
        x^{n + \alpha} +
        \sum_{n = 0}^{\infty}(x^2 - \nu^2)y_n x^{n + \alpha} = \\
        \sum_{n = 0}^{\infty}\left[\left[(n + \alpha)(n + \alpha - 1) + (n + \alpha) - \nu^2 \right]y_n + y_{n -
        2}\right]x^{n + \alpha} = 0
    \end{multline*}
    При $n = 0: y_0(\alpha^2 - \nu^2) = 0$. Поскольку мы считаем, что $y_0 \neq 0$, то $\alpha = \pm \nu$. Пусть скажем,
    $\alpha = \nu > 0$

    При $n = 1: ((\alpha + 1)^2 - \nu^2)y_1 = 0$. Поскольку $\alpha = \nu$, то $y_1 = 0$

    При $n = 2: ((\alpha + 2)^2 - \nu^2)y_2 + y_0 = 0$

    $\forall k \in \mathbb{N}: y_{2k + 1} = 0, y_{2k} = -\frac{y_{2k - 2}}{(2k + \alpha)^2 - \nu^2} =
    -\frac{y_{2k - 2}}{4k^2 + 4k\nu} = -\frac{y_{2k - 2}}{4k(k + \nu)}$

    Тогда $y_{2k} = (-1)^{k}\frac{y_0}{4^{k} k! \cdot (k + \nu) (k + \nu - 1) \cdots (\nu + 1)}$
\end{remark}

\begin{definition}
    $\Gamma(z) = \int_{-\infty}^{+\infty}t^{z - 1}e^{-t}dt$ -- Гамма-функция Эйлера. Причем $\Gamma(z+1) = z\Gamma(z)$,
    $\Gamma(n + 1) = n!$
\end{definition}

$y_{2k} = (-1)^k\frac{y_0 \Gamma(\nu + 1)}{4^{k}\Gamma(k + 1) \Gamma(k + \nu + 1)}$

Положим $y_0 = \frac{C}{2^{\nu}\Gamma(\nu + 1)}$

$y_{2k} = (-1)^k \frac{C}{2^{2k + \nu}\Gamma(k + 1)\Gamma(k + \nu + 1)}$

Рассмотрим отношение $\left| \frac{y_{2k + 2}}{y_{2k}}\right| = \frac{2^{2k + \nu}\Gamma(k + 1)\Gamma(k + \nu +
1)}{2^{2k + 2 + \nu}\Gamma(k + 2)\Gamma(k + \nu + 2)} = \frac{1}{4 (k + 1)(k + \nu + 1)} \rightarrow 0$ при $k
\rightarrow \infty$ при всех значениях $\nu$. Получаем, что $R = \infty$. Получили решение:

\[
    y_1(x) = Cx^{\nu}\sum_{k = 0}^{\infty} \frac{(-1)^k(\frac{x}{2})^{2k}}{\Gamma(k + 1) \Gamma(k + \nu + 1)} =
    J_{\nu}(x)
\]
называется функцией Бесселя. Если $\nu$ -- не целое, то есть второе решение
\[
    y_2(x) = C x^{-\nu}\sum_{k = 0}^{\infty}\frac{(-1)^k (\frac{x}{2})^{2k}}{\Gamma(k + 1) \Gamma (k - \nu + 1)} =
    J_{-\nu}(x)
\]. А если целое, то просто так выразить не удается и приходится прибегать к формуле Остроградского-Лиувилля
\end{document}
