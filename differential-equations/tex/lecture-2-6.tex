\documentclass[document.tex]{subfiles}

\begin{document}
\section{Теорема Штурма}
Будем рассматривать уравнения вида
\begin{equation}
    \label{sturm-eq}
    y''(x) + a(x)y' + b(x)y = 0
\end{equation}
, где $a(x)$ -- непрерывно дифференцируема на некотором
интервале $I$, а $b(x)$ -- непрерывна на этом же самом интервале.

\begin{definition}
    Решение $y_1$ уравнения \ref{sturm-eq} называется нетривиальным, если $\exists x \in I: y_1(x) \neq 0$
\end{definition}

\begin{definition}
    $x_0$ называется нулём решения $y_1$ уравнения \ref{sturm-eq}, если $y_1(x_0) = 0$
\end{definition}

\begin{definition}
    Если у решения уравнения \ref{sturm-eq} есть два нуля, то такое решение называется колеблющимся.
\end{definition}

Выполним замену, чтобы избавитсья от слагаемого $a(x)y'$. Положим $y(x) = u(x) z(x)$. Тогда $y' = u'z + uz'$, $y'' =
u''z + 2u'z' + uz''$. В рамках этой замены уравнение \ref{sturm-eq} переписывается в виде:
\[
    u''z + 2u'z' + uz'' + a(u'z + uz') + buz = 0
\]
Группируя слагаемые по порядку производной $z$, получаем:
\[
    uz'' + (2u' + au)z' + (u'' + au' + bu)z = 0
\]
Мы хотим, чтобы $2u' + au = 0$. Решая это дифференциальное уравнения относительно $u$, имеем: $\ln |u| =
-\frac{1}{2}\int_{x_0}^{x}a(t)dt$, иначе $u = \exp \left[ -\frac{1}{2}\int_{x_0}^{x}a(t)dt \right]$. $u' =
-\frac{1}{2} a u$, $u'' = -\frac{1}{2}(a'u + u'a) = -\frac{1}{2}(a'u - \frac{1}{2}u a^2) =
\frac{1}{2}u(\frac{1}{2}a^2 - a')$. Тогда уравнение переписывается в виде:
\[
    uz'' + u(\frac{a^2}{4} - \frac{a'}{2} - \frac{a^2}{2} + b)z = 0
\]
Сокращая на $u \neq 0$, получаем
\[
    z'' + (b - \frac{a^2}{4} - \frac{a'}{2})z = 0
\]
Причем, в силу того, что $u > 0$, у $z$ будет ноль в точке $x_0$ тогда и только тогда, когда у $y$ будет ноль в точке
$z_0$.

Поэтому перейдем к рассмотрению уравнения
\begin{equation}
    \label{sturm}
    z'' + q(x)z = 0
\end{equation}

\begin{definition}
    $x_0$ называется простым нулём решения $z$ уравнения \ref{sturm}, если он является нулём и дополнительно $z'(x_0) \neq 0$
\end{definition}

\begin{lemma}
    \label{good_lemma}
    Все нули нетривиального решения $z_1$ уравнения \ref{sturm} являются простыми.
\end{lemma}

\begin{proof}
    Предположим противное. Пусть у нетривиального решения $z_1$ уравнения \ref{sturm} если такой ноль $x_0$, что $z_1'(x_0) = 0$. Тогда
    поставим задачу Коши следующим образом:
    \[
        \begin{cases}
            z'' + q(x)z = 0, \\
            z(x_0) = 0, \\
            z'(x_0) = 0, \\
        \end{cases}
    \]
    Тогда нетривиальное решение $z_1$ является решением данной задачи Коши. Но и тривиальное решение $z = 0$ также
    является решением данной задачи Коши. Мы знаем, что решение задачи Коши единственно. Значит $z_1 = 0$ на $I$. Что
    противоречит тому, что $z_1$ -- нетривиальное решение.
\end{proof}

\begin{lemma}
    Нули любого предельного решения уравнения \ref{sturm} не имеют предельной точки в $I$.
\end{lemma}

\begin{proof}
    Пусть $z_1$ -- решение уравнения \ref{sturm}, пусть множество $N = \{x \in I: z_1(x) = 0\}$ -- нули этого решения. Пусть у
    этого множества есть предельная точка в $I$. Тогда существует последовательность $\{x_k \in N\}_{k = 1}^{\infty}$,
    такая что $x_k \rightarrow x_0 \in I$, причем $\forall k: z_1(x_k) = 0$.
    $z_1$ непрерывна как решение дифференциального уравнения. Из определения непрерывности функции и определения предела
    функции по Гейне следует, что
    $0 = \lim_{k \rightarrow \infty} z_1(x_k) = z_1(x_0)$.
    Рассмотрим функцию
    \[
        h(x) = \frac{z_1(x) - z_1(x_0)}{x - x_0}
    \]. Она непрерывна везде кроме $x_0$ как композиция непрерывных функций и в точке $x_0$, т.к. $z_1$ дифференцируема
    в этой точке (как решение дифференциального уравнения). Раз она непрерывна, то из определения непрерывности по Гейне
    \[
        z_1'(x_0) = h(x_0) = \lim_{k \rightarrow \infty} \frac{z_1(x_k) - z_1(x_0)}{x_k - x_0} = \lim_{k \rightarrow
        \infty} \frac{0 - 0}{x_k - x_0} = 0
    \]
    Значит, $x_0$ не является простым нулём нетривиального решения $z_1$. Но это противоречит предыдущей лемме.
\end{proof}

\begin{corollary}
    Любое нетривиальное решение имеет на всяком отрезоке $[\alpha, \beta] \subset I$ лишь конечное число нулей.
\end{corollary}

\begin{proof}
    Если это не так, то из множества нулей по теореме Больцано-Вейерштрасса можно выделить сходящуюся
    подпоследовательность. Так как $[\alpha, \beta]$ -- компакт, то предел этой последовательности будет лежать в
    $[\alpha, \beta]$, а значит в $I$. Тогда это противоречит предыдущей лемме.
\end{proof}

\begin{theorem}[Штурм]
    Рассмотрим на некотором промежутке $I$ два уравнения:
    \begin{equation}
        \label{eq1}
        y'' + q(x)y = 0
    \end{equation}
    \begin{equation}
        \label{eq2}
        z'' + Q(x)z = 0
    \end{equation}, где $q(x), Q(x)$ непрерывны на $I$, причем $\forall x \in I: q(x) \geq Q(x)$. Обозначим за $y(x)$
    некоторое нетривиальное решение уравнения \ref{eq1}, а за $z(x)$ некоторое нетривиальное решение уравнения \ref{eq2}
    Пусть $x_1, x_2 \in I$ -- последовательные нули решения $y(x)$. Тогда либо $z(x_1) = z(x_2) = 0$, либо $\exists x_0
    \in (x_1, x_2): z(x_0) = 0$
\end{theorem}

\begin{proof}
    Пусть без ограничения общности $\forall x \in (x_1, x_2): y(x) > 0$. Так как решение дифференциального уравнения по
    определению непрерывно дифференцируемо, то правая и левая производная $y$ в любой точке совпадают. Тогда
    \[
        y'(x_1) = \lim_{x \rightarrow x_1 + 0}
        \frac{y(x) - y(x_1)}{x - x_1} = \lim_{x \rightarrow  x_1 + 0} \frac{y(x)}{x - x_1} \geq 0
    \]
    \[
        y'(x_2) = \lim_{x \rightarrow x_2 - 0} \frac{y(x) - y(x_2)}{x - x_2} = \lim_{x \rightarrow x_2 - 0}
        \frac{y(x)}{x - x_2} \leq 0
    \]
    Из леммы \ref{good_lemma} мы знаем, что $y'(x_1) \neq 0$, $y'(x_2) \neq 0$. Тогда $y'(x_1) > 0$, $y'(x_2) < 0$. Мы
    знаем что для решений $y$, $z$ выполняется:
    $y'' + q(x)y = 0$, $z'' + Q(x)z = 0$. Тогда $y''z + q(x)yz = 0$, $z''y + Q(x)yz = 0$.
    Тогда
    \[
        (q(x) - Q(x))yz = y''z - z''y = y''z + y'z' - z''y - y'z' = (y'z - z'y)'
    \].
    Проинтегрируем это уравнение от $x_1$ до $x_2$.
    Тогда
    \begin{multline*}
        \int_{x_1}^{x_2}(q(x) - Q(x))y(x)z(x)dx = y'(x)z(x) - z'(x)y(x) \Big|_{x_1}^{x_2} = y'(x_2)z(x_2) -\\
        y'(x_1)z(x_1) + y(x_1)z'(x_1) - y(x_2)z'(x_2) = y'(x_2)z(x_2) - y'(x_1)z(x_1)
    \end{multline*}
    Если $\forall x \in (x_1, x_2): z(x) \neq 0$, то либо $z(x) > 0$, либо $z(x) < 0$. Пусть без ограничения общности
    $z(x) > 0$, а также либо $z(x_1) \neq 0$, либо $z(x_2) \neq 0$. Тогда подинтегральная функция неотрицательна (а
    значит и интеграл), а правая часть уравнения отрицательна. Противоречие.
\end{proof}

\end{document}
