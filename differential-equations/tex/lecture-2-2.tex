\documentclass[document.tex]{subfiles}

\begin{document}

\section{Банаховы пространства. Теорема Банаха}

\begin{definition}
Нормой $||x||$ на линейном пространствен называется функция $||x|| : V \mapsto \mathbb{R}$, такая, что:
	\begin{enumerate}
		\item $\forall x : ||x|| > 0, ||x|| = 0 \Leftrightarrow x = 0$
		\item $\forall x, \lambda : ||\lambda x|| = |\lambda| ||x||$
		\item $\forall x, y : ||x + y|| \leq ||x|| + ||y||$
	\end{enumerate}
\end{definition}

\begin{definition}
	Последовательность $\{x_n\}$ называется сходящейся, если $\exists x \in \mathbb{L}: \lim_{n \rightarrow \infty} ||x_n - x|| = 0$
\end{definition}

\begin{definition}
	Фундаментальная последовательность определяется аналогично
\end{definition}

\begin{definition}
	Нормированное пространство, в котором каждая фундаментальная последовательность является сходящейся, называется полным (банаховы)
\end{definition}

\begin{definition}
	Отображение $\Phi : X \subset \mathbb{L}_1 \mapsto \mathcal{L}_2$ называется непрерывным в точке $x_0 \in X$:
	Аналогичное
	
\end{definition}

\begin{definition}
	Точка $x^*$ называется неподвижной точкой отображения $\varphi : X \subset \mathcal{L} \mapsto \mathcal{L}$, если $\varphi(x^*) = x^*$.
\end{definition}

\begin{definition}
	Отображение $\varphi$ назывется сжимающим, если $\exists q : ||\varphi(x_1) - \varphi(x_2)|| < q ||x_1 - x_2||$
\end{definition}

\begin{theorem}[Принцип сжимающих отображений, теорма Банаха]
	Пусть замкнутое $U_r(x_0) \subset \mathcal{L}$, $\varphi$ является сжимающим на $U_r(x_0)$ с коэффициентом $q$. Тогда, если выполнено условие
	$||\varphi(x_0) - x_0|| \leq (1 - q)r$, то отображение $\varphi$ имеет единственную неподвижную точку.
\end{theorem}

\begin{proof}
	Докажем сначала, что шар отображается сам в себя: рассмотрим $x \in U_r(x_0)$:
	$$||\varphi(x) - x_0|| = ||\varphi(x) - \varphi(x_0) + \varphi(x_0) - x_0|| \leq ||\varphi(x) - \varphi(x_0)|| + ||\varphi(x_0) - x_0||$$
	$$q||x - x_0|| + (1-q)r \leq qr + (1-q)r = r$$
	Мы доказали, что образ шара -- это шар. Рассмотрим реккурентную последовательность $x_n = \varphi(x_{n-1})$.
	$$||x_n - x_m|| = ||x_{n+p} - x_n|| = ||x_{n+p} - x_{n+p-1} + x_{n+p-1} - x_{n+p-2} + ...|| \leq \sum_{i = 0}^p ||x_{n+i} - x_{n+i-1}||$$
	$$||x_2 - x_1|| = ||\varphi(x_1) - \varphi(x_0)|| \leq q||x_1 - x_0||$$
	Проводя аналогичные рассуждения, имеем:
	$$||x_n - x_{n-1}|| = q^{n-1}||\varphi(x_0) - x_0||$$
	Суммируя это, получаем:
	$$q^{n+p-1} l + ... + q^n l = q^n l \frac{q^{p} - 1}{q-1}$$
	Значит, эта последовательность является фундаментальной, существует предел $x^*$ и так как шар замкнут, то предел принадeжит шару.
	Заметим, что $\varphi$ является равномерно непрерывной. Кроме того, 
	$$x^* = \lim_{n \rightarrow \infty} x_n = \lim_{n \rightarrow \infty} \varphi(x_n) = \varphi(x^*)$$.
	Докажем единственность. Пусть $\exists x^O : \varphi(x^O) = x^O$. Рассмотрим норму разности между ними:
	$$||x^O - x^*|| = ||\varphi(x^O) - \varphi(x^*)|| \leq q ||x^O - x^*||$$.
\end{proof}

\begin{theorem}[о существовании и единственности решения задачи Коши]
	Рассмотрим область $G \subset \mathbb{R}^{n+1}$. Пусть вектор функция $f(x, y)$ удовлетворяет условию Липшица на любом компакте в $G$ 
	по переменной $y$ равномерно по $x$. И пусть $(x_0, y_0) \in G$. Тогда:
	\begin{enumerate}
		\item $\exists \delta > 0 : \exists y$ определенная на $[x_0 - \delta, x_0 + \delta]$: $y$ является решением задачи Коши.
		\item Решение задачи Коши единственно в том смысле, что если $y_1$ является решением задачи Коши на отрезке $[x_0 - \delta_1,
		x_0 + \delta_1]$, а $y_2$ решением задачи Коши на отрезке $[x_0 - \delta_2, x_0 + \delta_2]$, то их огранечения на наименьший из
		отрезков тождественно равны
	\end{enumerate}
	
	
\end{theorem}

\begin{proof}
	$G$ -- область, следовательно любая точка $(x, y)$ принадлежит вместе со своей окрестностью, в том числе и замыкание некоторой окрестности
	$U(x, y)$. Заметим, что все $f_i$ непрерывны и ограничены.
	Рассмотрим норму $$||f|| = \max_{1 \leq i \leq n} \sup_{(x, y)} f_i(x, y)$$. Вложим в каждый замкнутый шар цилиндр: 
	$$T_{r'}(x, y) = \{(x, y) \in U(x, y) : x \in [x_0 - \delta, x_0 + \delta], ||y - y_0|| \leq r\}$$
	При этом выберем $r'$ и $\delta$ соответственно, чтобы цилиндр лежал внутри шара.
	Рассмотрим уравнение $$y(x) = y_0 + \int_{x_0}^x f(\tau, y(\tau))d\tau$$ для которого мы решаем задачу Коши. Рассмотрим оператор $\Phi$
	действующий из пространства функций на шаре в себя, такой что:
	$$\Phi (y) (x) = y_0 + \int_{x_0}^x f(\tau, y(\tau)) d\tau$$
	Докажем, что этот оператор опять сжимает. Пусть y и z -- две различные вектор функции.
	$$||\Phi(y) - \Phi(z)|| = \max \sup_{[x_0 - \delta_{r'}, x_0 + \delta_{r'}]} | \int_{x_0}^x (f_i(\tau, y(\tau) - f_i(\tau, z(\tau)))d\tau|$$
	$$\leq \sup \int_{x_0}^x ||f(\tau, y) - f(\tau, z)||d\tau \leq \sup \int_{x_0}^x c ||y-z||d\tau \leq \delta c||y-z||$$
\end{proof}
\end{document}
