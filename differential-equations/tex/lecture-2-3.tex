\documentclass[document.tex]{subfiles}

\begin{document}
\section{Продолжения решений задачи Коши}

\begin{definition}
	Пусть $x$ -- точка, $G$ -- множество. Тогда
	$$\rho(x, G) = \inf_{x' \in G} \|x - x'\|$$
	-- расстояние от множества $G$ до точки $x$
\end{definition}

\begin{definition}
	Если $G, F$ -- множества, то
	$$\rho(G, F) = \inf_{x_1 \in G, x_2 \in F} \|x_1 - x_2\|$$
	-- расстояние между множествами $G$ и $F$
\end{definition}

\begin{theorem}
	Пусть вектор-функция $f(x, y)$ удовлетворяет условиям теормы о существовании и единственности решения задачи Коши в некоторой замкнутой области $\overline G \subset \mathbb{R}^{n+1}$. Тогда любое решение задачи Коши $y(x): y' = f(x, y)$, интегральная кривая которого проходит через точку $(x_0, y_0)$ области $G$, можно продолжить в обе стороны от $x_0$ вплоть до выхода на границу $\gamma = \partial G$, то есть можно продолжить $y(x)$ на $[a, b]$ так, что $(a, y(a)), (b, y(b)) \in \gamma$
\end{theorem}

\begin{proof}
	Рассмотрим
	$$T_r = \{(x, y): \cdots \}$$
	$r, \delta_r = \frac{r}{M + Kr}$ из предыдущщего доказательства.

	Рассмотрим точку $P_0 = (x_0, y_0)$ через которую проходит решение задачи Коши. $T_0 : min(\delta_0, r_0) = \rho(P_0, \gamma)$
	На интервале $I_0 = [x_0 - \delta_0, x_0 + \delta_0]$ есть решение задачи Коши. Обозначим $x_0 + \delta_0 = x_1, y(x_1) = y_1$. Рассмотрим точку $P_1 = (x_1, y_1)$. Поставим  новую задачу Коши с точкой $(x_1, y_1)$. Построим новый циллиндр $T_1 : min(\delta_1, r_1) = \rho(P_1, \gamma)$. На новом промежутке $I_1 = [x_1 - \delta_1, x_1 + \delta_1]$ тоже есть решение задачи Коши, причем эти два решения совпадают на $I_1 \cap I_2$. Действительно, $x_1 \in I_1, I_2$ и на $x_1$ оба решения совпадают. Значит, они совпадают на $I_1 \cap I_2$. Проделываем такие рассуждения счетное число раз. Тогда $\delta_k \rightarrow 0$, а значит, $r_k \rightarrow 0$. т.е. радиусы шаров стремятся к нулю.

	Теперь докажем, что $x_k$ сходятся, куда надо, то есть к границе $\gamma$. Пусть $b = \lim_{k \rightarrow \infty} x_k$. Понятно, что этот предел существует, поскольку последовательность ограничена и монотонна. Зафиксируем $\varepsilon > 0$. Рассмотрим $\alpha, \beta \in (b - \varepsilon, b) \subset [x_0, b)$:
	$$\|y(\beta) - y(\alpha)\| = \|\int_{\alpha}^{\beta}f(\tau, y(\tau)) d\tau \| \leq M(\beta - \alpha) \leq M \varepsilon$$
	Используя критерий Коши, имеем: $\lim_{x \rightarrow b - 0} y(x) = y^*$. Так как $y$ непрерывна, $y(b) = y^*$. Положим $P^* = (b, y(b))$. Докажем, что $P^* \in \gamma$. Допустим, что $P^* \neq \gamma$. Тогда $U(P^*) \subset G$. Тогда $\exists \varepsilon > 0: U_{2 \varepsilon}(P^*) \subset G$. Тогда $\rho(P^*, \gamma) \geq 2 \varepsilon$. Причем $P_n \rightarrow P^*$. То есть $\forall \varepsilon' > 0 : \exists k_{\varepsilon'} \in \mathbb{N}: P_k \subset U_{\varepsilon'}(P^*)$. Значит, $\forall k' > k_{\varepsilon}: \rho(P_k, \gamma) > \varepsilon$. Но тогда неверно, что $r_k \rightarrow 0$

	Таким образом, мы получили два продолжения решения задачи Коши на область $G$.
\end{proof}

\begin{corollary}
	Пусть $G$ -- неограниченное замкнутое связное множество из $\mathbb{R}^{n+1}$ такое, что $\forall c, d:$ часть множества $G_{cd} = G \cap \{x : c \leq x \leq d\}$ ограниченна. Пусть в $G$ вектор функция $f(x, y)$ удовлетворяет условию теоремы о существовании и единственности задачи Коши. Тогда решение $y(x)$, интегральная кривая которого проходит через току $(x_0, y_0)$, продолжается в каждую сторону или до выхода интегральной кривой на границу $\gamma = \partial G$, либо неограниченно по $x$, то есть до сколь угодно большого значения $|x|$
\end{corollary}

\begin{proof}
    Рассмотрим последовательность $d_i \rightarrow \infty$. Будем последовательно строить решение задачи Коши для
    отрезков $[c, d_i]$ как в предыдущем доказательстве. Аналогично, либо мы смогли построить решение на бесконечном
    отрезке, либо когда-нибудь вышли на границу.
\end{proof}


\end{document}
