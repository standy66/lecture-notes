\documentclass[document.tex]{subfiles}

\begin{document}
\section{Дифференциальные уравнения в частных производных}
Рассмотрим $F(x_1, \cdots, x_n, u, p_1, \cdots, p_n)$ -- непрерывно дифференцируемую на $G \subset \mathbb{R}^{2n + 1}$
для которой выполняется
\[
    \sum_{i = 1}^{n} \left( \frac{\partial F}{\partial p_i} \right)^2 \neq 0
\]
на всей области $G$.

\begin{definition}
    Уравнение 
    \begin{equation}
        \label{partial-eq}
        F(x_1, \cdots, x_n, u, u'_{x_1}, \cdots, u'_{x_n}) = 0
    \end{equation}
    называется уравнением в частных производных относительно $u(x) = u(x_1, \cdots, x_n)$
\end{definition}

\begin{definition}
    Функция $u = \varphi(x_1, \cdots, x_n)$ заданная в области $\Omega \subset \mathbb{R}^n$ называется решением
    уравнения \ref{partial-eq}, если выполнены следующие условия:
    \begin{enumerate}
        \item $\varphi$ -- непрерывно-дифференцируема
        \item $\forall (x_1, \cdots, x_n) \in \Omega: (x_1, \cdots, x_n, \varphi(x_1, \cdots, x_n), \varphi'_{x_1},
            \cdots, \varphi'_{x_n}) \in G$
        \item $\forall (x_1, \cdots, x_n) \in \Omega: F(x_1, \cdots, x_n, \varphi(x_1, \cdots, x_n), \varphi'_{x_1},
            \cdots, \varphi'_{x_n}) = 0$
    \end{enumerate}
\end{definition}

\begin{definition}
    Уравнение \ref{partial-eq} называется линейным, если $u$ и все частные производные входят в систему линейным
    образом, то есть система имеет вид:
    \begin{equation}
        \label{part-lin}
        (\nabla u, a(x)) + b(x)u = f(x)
    \end{equation}
\end{definition}

\begin{definition}
    \begin{equation}
        \label{hom-lin}
        (\nabla u, a(x)) = 0
    \end{equation}
    -- линейное однородное уравнение
\end{definition}

\begin{definition}
    Автономная система уравнений
    \[
        \dot x = a(x)
    \] задаваемая в области $\Omega$ называется характерестической системой уравнений для уравнения \ref{hom-lin}, а её
    траектории называются характеристиками уравнения \ref{hom-lin}
\end{definition}

\begin{theorem}[об общем решении однородного линейного дифференциального уравнения в частных производных]
    $\forall b \in \Omega: \exists U(b):$ общее решение уравнения \ref{hom-lin} в этой окрестности имеет вид: $u(x) =
    f(u_1(x), \cdots, u_{n - 1}(x))$, где $u_1(x), \cdots, u_{n - 1}(x)$ -- независимые в точке $b$ первые интегралы
    характерестической системы, а $f$ -- произвольная непрерывно-дифференцируемая функция. То есть для любого решения в
    некоторой окрестности существует такая $f$, и для любой непрерывно-дифференцируемой $f$ $u(x)$ будет является
    решением.
\end{theorem}
\begin{proof}
    Левая часть уравненния \ref{hom-lin} -- это производная от функции $u(x)$ в силу характерестической системы. То есть
    любое решением уравнения -- это первый интеграл системы. Так как $\forall x \in \Omega$ имеет место условие $\sum_{j
    = 1}^{n}a_j(x)^2 \neq 0$, то в области $\Omega$ нет положения равновесия характерестической системы. Следовательно,
    для любой точки $b \in \Omega$ $\exists U(b)$ в которой имеются $n - 1$ независимый первый интеграл. А значит любой
    первый интеграл характересической системы выражается как функция от независимых первых интегралов.
\end{proof}

\subsection{Начальные условия для уравнения \ref{hom-lin}}

\begin{definition}
    Пусть $g(x) = 0$ задает некоторую гиперповерхность $\gamma$. Это кривая называется начальной гиперповерхностью.
    Система
    \begin{equation}
        \label{cauchy_part}
        \begin{cases}
            (\nabla u, a(x)) = 0, \\
            u(x) \Big|_{x \in \gamma} = \varphi(x)
        \end{cases}
    \end{equation} 
    называется задачей Коши с начальной гиперповерхностью $\gamma$
\end{definition}

\begin{definition}
    Решением задачи Коши \ref{cauchy_part} называется функция $u(x)$ удовлетворяющая уравнению \ref{hom-lin}, такая что
    $u \Big|_{\gamma} \equiv \varphi$
\end{definition}

\begin{definition}
    Точка $M \in \gamma$ в которой $(\nabla g, a) = 0$, то $M$ называется характеристической точкой уравнения \ref{hom-lin}
\end{definition}

\begin{theorem}[о решении задачи Коши для линейного однородного дифференциального уравнения в частных производных]
    Если $M_0 \in \gamma$ не является характерестической точкой уравнения \ref{hom-lin}, то решение задачи Коши
    \ref{cauchy-part} существует и единственно
\end{theorem}

\begin{proof}
    Так как $a(M_0) \neq 0$, то существует окресность точки $M_0$ в которой имеется $n - 1$ независимый первый интеграл
    $u_1(x), \cdots, u_{n - 1}(x)$
    харакетрестической системы уравнений. Известно, что в некоторой окрестности этой точки любое решение представляетя в
    виде:
    \[
        u(x) = f(u_1(x), \cdots, u_{n - 1}(x))
    \] Покажем, что начальные условия однозначно определяют функцию $f$. Рассмотрим в окрестности $M_0$ следующую
    систему уравнений:
    \[
        \label{temp_system_proof}
        \begin{cases}
            u_1(x) = u_1, \\
            \cdots, \\
            u_{n - 1}(x) = u_{n - 1}, \\
            g(x) = 0
        \end{cases}
    \] Покажем, что в окресности точки $M_0$ выполнены условия теоремы о неявной функции. Рассмотрим матрицу Якоби этой
    системы в точке $M_0$:

    Допустим, что Якобиан равен нулю в точке $M_0$. Значит строки матрицы Якоби линейно зависимы. Но первые $n - 1$
    строчка -- это первые интегралы, которые по условию независимы. Значит, $\exists \{c_k\}: g'_{x_j} = \sum_{i = 1}^{n
    - 1} c_i \frac{\partial u_i}{\partial x_j}$. Значит $(\nabla g, a) = \sum_{j = 1}^n a_j \sum_{k = 1}^{n - 1} c_k
    \frac{\partial u_k}{\partial x_j} = \sum_{k = 1}^{n - 1} c_k \sum_{j = 1}^n a_j \frac{\partial u_k}{\partial x_j} =
    0$. Получили противоречие. Значит, система уравнений \ref{temp_system_proof} допускает единственное выражение $x =
    \omega(u_1, \cdots, u_{n - 1})$. Тогда $\varphi(x) = \varphi(\omega(u_1, \cdots, u_{n - 1}))$ является решением
    исходного уравнения, причем она единственна по построению.
\end{proof}

\section{Основы вариационного исчисления}
\begin{example}[задача Дидоны]
    Задача Дидоны — исторически первая задача вариационного исчисления. Связана с древней легендой об основании города Карфагена. Дидона — сестра царя финикийского города Тира, переселилась на южное побережье Средиземного моря, где попросила у местного племени участок земли, который можно охватить шкурой быка. Местные жители предоставили шкуру, которую Дидона разрезала на узкие ремни и связала их. Получившимся канатом охватила территорию у побережья. Возникает вопрос о том, как можно захватить максимальную площадь.
\end{example}

\begin{example}[задача о Брахистохроне]
    Среди плоских кривых, соединяющих две данные точки А и В, лежащих в одной вертикальной плоскости (В ниже А), найти ту, двигаясь по которой под действием только силы тяжести, сонаправленной отрицательной полуоси OY, материальная точка достигнет В из А за кратчайшее время
\end{example}

\begin{definition}
    Функция $\rho(x, y)$ двух переменных $x, y$, лежащих в некотором пространстве $X$ называется метрикой, если $\forall
    x \neq y, z$:
    \begin{enumerate}
        \item $\rho(x, y) = \rho(y, x)$
        \item $\rho(x, x) = 0$, $\rho(x, y) > 0$
        \item $\rho(x, z) \leq \rho(x, y) + \rho(y, z)$
    \end{enumerate}
\end{definition}

\begin{remark}
    Соответсвующее пространство с введённой метрикой называется метрическим. В Банаховых пространствах можно ввести
    метрику как $\rho(x, y) = \|x - y\|$. В частности, в прастранстве $C^{(k)}[a, b]$ (пространство $k$ раз
    непрерывно-дифференцируемой) $\|\varphi(x)\| = \sum_{i = 1}^k \max_{[a, b]}|\varphi^{(i)}(x)|$
\end{remark}

\begin{definition}
    Рассмотрим $M \subset Y$, где $Y$ -- метрическое пространство. Отображение $F: M \mapsto \mathbb{R}$ называется
    функционалом с областью определения $M$
\end{definition}

\begin{remark}
    Рассмотрим пространство $C[a, b]$. Введём в нём метрику $\|y_1(x) - y_2(x)\| = \max_{[a, b]}|y_1(x) - y_2(x)| +
    \max_{[a, b]} |y'_1(x) - y'_2(x)|$. Такая метрика традиционно называется слабой, потому что она сильно ограничивает
    понятие близких функций.

    Пусть $F(x, y, p)$ -- непрерывно-диффиринцируема по $x$ на $[a, b]$ при любом фиксированном $(y, p) \in
    \mathbb{R}^2$. Мы будем рассматривать функционал
    \[
        J(y) = \int_{a}^{b}F(x, y, y')dx
    \]. Потребуем, чтобы $y(a) = A, y(b) = B$. Все такие непрерывно-дифференцируемые $y$ назовём допустимыми.
\end{remark}

\begin{definition}
    Говорят, что функция $\hat y$, являющаяся допустимой, доставляет (дает) слабый экстремум (максимум или минимум)
    функционалу $J(y)$, если $\exists \varepsilon > 0: \forall y \in U_{\varepsilon}(\hat y): J(y) \leq J(\hat y)$.
    Либо, в случае минимума $J(y) \geq J(\hat y)$
\end{definition}

\begin{definition}
    Задачу нахождения слабого экстремума $J(y)$ называется задачей с закреплёныыми концами, или простейшая задача
    вариационного исчисления.
\end{definition}

\subsection{Вариация функционала}

\begin{definition}
    Обозначим $\circ C^{1}[a, b]$ пространство функций $C^{1}[a, b]$ таких, что $y(a) = y(b) = 0$. Пусть $y(x) \in M$
    ($M$ -- пространство допустимых функций), $\eta(x) \in \circ C^1[a, b]$. Положим $\forall \alpha \in
    \mathbb{R}: y_{\alpha}(x) := y(x) + \alpha \eta(x)$
\end{definition}

\begin{lemma}[основная лемма вариационного исчисления]
    Если $f(x) \in C[a, b]$ и $\int_{a}^{b}f(x)\eta(x)dx = 0$ $\forall \eta \in \mathring{C}^1[a, b]$, то $f \equiv 0$
    на $[a, b]$ 
\end{lemma}

\end{document}
