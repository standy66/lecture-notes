\documentclass[document.tex]{subfiles}

\begin{document}
\section{Сходимость ряда Фурье.}
\begin{theorem}
	Если $f$ -- $2 \pi$ периодическая и абсолютно интегрируемая на $[-\pi, \pi]$. Пусть $x_0$ -- почти регулярная точка. Тогда
	$$S_n(x_0) \rightarrow \frac{f(x_0 - 0) + f(x_0 + 0}{2}$$
\end{theorem}
\begin{proof}
	$S_n(x) = \int_{0}^{\pi} D_n(t) (f(x-t) + f(x+t)) dt$, где $D_n(t) = \frac{\sin (n+\frac{1}{2}) t}{2 \sin \frac{t}{2}}$.
	Пусть $\delta \in (0, \pi)$. Посчитаем разноcть
	\begin{multline*}
		S_n(t) - \frac{f(x_0 - 0) + f(x_0 + 0)}{2} = \frac{1}{\pi}\int_{0}^{\pi}D_n(t)(f(x_0 - t) + f(x_0 + t)) dt -\\
		\frac{1}{\pi} \int_{0}^{\pi}D_n(t) (f(x_0 - 0) - f(x_0 + 0)) = \frac{1}{\pi} \int_{0}^{\pi} (f(x_0 - t) - f(x_0 - 0) + f(x_0 + t) + f(x_0 + 0))\\
		\frac{\sin (n + 1/2)t}{\sin t/2} dt = \frac{1}{\pi} \left( \int_{0}^{\delta} + \int_{\delta}^{\pi} \right)  \frac{f(x_0 - t) + f(x_0 - 0)}{t} + \\
		\frac{f(x_0 + t) + f(x_0 + 0)}{t} \frac{t}{2 \sin(t/2)} \sin (n + 1/2) t dt
	\end{multline*}
	Пусть $\varepsilon > 0$. $\exists \delta' \in (0, \delta) : |\frac{f(x_0 - t) + f(x_0 - 0)}{t}| \leq |f'_-(x_0)| + \varepsilon$. Аналогично для второй дроби. Поэтому первый интеграл не превосходит $\frac{1}{\pi} \delta (|f'_-| + |f'_+| + 2 \varepsilon)$. На интервале $(\delta, \pi)$ подынтегральная функция абсолютно интегрируема, поэтому по лемме Римана об осцилляции второй интеграл стремится к 0.
\end{proof}

\begin{remark}[Условие Гёльдера]
	Можно потребовать, чтобы для некоторого $\alpha > 1$ было выполнено $|f(x_0 + t) - f(x_0 + 0)| \leq C \cdot t^{\alpha}$, $|f(x_0 - t) + f(x_0 - 0)| \leq C \cdot t^{\alpha}$. Это является более слабым условием по сравнению с сущетвованием односторонних производных, однако, при этом условии теорема доказывается так же.
\end{remark}

\begin{definition}
	Функция $f$ называется кусочно-гладкой на $[a, b]$, если она непрерывна на $[a, b]$ и $\exists \text{ } a = c_0 <c_1 < \cdots < c_n = b:$ для любого отрезка $[c_{k-1}, c_k]$ функция $f$  непрерывно дифференцируема на нём
\end{definition}

\begin{definition}
	$2 \pi$ периодическая функция называется кусочно-гладкой, если она непрерывна на $\mathbb{R}$ и кусочно-гладкая на $[-\pi, \pi]$
\end{definition}

\begin{theorem}
	Пусть $f$ кусочно-гладкая $2 \pi$ периодическая функция. Тогда:
	\begin{enumerate}
		\item $S_n(f) \rightarrow f$ равномерно
		\item $\sup_{x \in \mathbb{R}} |S_n(f, x) - f(x)| \leq C \cdot \frac{\ln n}{n}$
	\end{enumerate}
\end{theorem}

\begin{proof}
	\begin{multline*}
		S_n(f, x) - f(x) = \frac{1}{\pi} \left( \int_{0}^{\delta} + \int_{\delta}^{\pi} \right) D_n(t) (f(x+t) + f(x-t) - 2f(x))dx = \\
		\frac{1}{\pi} \left( \int_{0}^{\delta} + \int_{\delta}^{\pi} \right) \sin ((n + 1/2)t) g_x(t) dt
	\end{multline*}
	Оценим с помощью теоремы Лагранжа следующую разность: $|f(x+t) + f(x-t) - 2f(x)| \leq |f(x+t) - f(x)| + |f(x - t) - f(x)| \leq 2 |t| M_1 \leq 2 \pi M_1$. Кроме того, оценим
	$|\frac{d}{dt} g_x(t)| \leq \cdots$

	Причем, $|\int_{0}^{\delta} |$
    TO BE CONTINUED\ldots
\end{proof}

\begin{definition}
	Функция
	$$T_n(x) = \frac{a_0}{2} + \sum_{k = 1}^n a_k \cos kx + b_k \sin kx$$
	называется тригонометррическим многочленом порядка $n$
\end{definition}

\begin{theorem}
	Пусть $f$ -- непрерывная $2 \pi$ периодическая функцияю. Тогда $\forall \varepsilon > 0 : \exists T(x) : \sup_{x \in \mathbb{R}} |T(x) - f(x)| < \varepsilon$
\end{theorem}

\begin{proof}

\end{proof}

\end{document}

