\documentclass[document.tex]{subfiles}

\begin{document}
\section{Равномерная функциональная сходимость на множестве}
\begin{definition}
    Пусть $X, Y \subset \mathbb{R}$, $y_0 \in Y$ -- предельная точка $Y$. Пусть заданы функции $f : X \times Y \mapsto
    \mathbb{R}, \varphi: X \mapsto \mathbb{R}$. Говорят, что функция $f$ равномерно на $X$ стемится к $\varphi$ при $y
    \rightarrow y_0$ и пишут $f \rightrightarrows \varphi$ на $X$ при $y \rightarrow y_0$, если
    \[
        \sup_{x \in X}|f(x, y) - \varphi(x)| \rightarrow 0 \text{ при } y \rightarrow y_0
    \]
\end{definition}

\begin{remark}
    Понятие равномерной сходимости по множеству обобщает понятие равномерной сходимости функциональной
    последовательности. Действительно, пусть $Y = \mathbb{N}$. $f_n(x) = f(x, n), y_0 = +\infty$. Тогда утверждение $f(x, n)
    \rightrightarrows \varphi(x)$ на $X$ при $y \rightarrow +\infty$ совпадает с утверждением $f_n(x) \rightrightarrows
    \varphi(x)$ при $n \rightarrow +\infty$
\end{remark}

\begin{theorem}[критерий Коши]
    Для того, чтобы заданная на $X \times Y \subset \mathbb{R}^2$ функция $f$ равномерно по $X$ стремилась к какой-либо
    функции при $y \rightarrow y_0$ необходимо и достаточно выполнения условия Коши:
    \[
        \forall \varepsilon > 0: \exists \delta: \forall y', y'' \in Y \cap \mathring{U}_{\delta}(y_0):
        \sup_{X} |f(x, y') - f(x, y'')| < \varepsilon
    \]
\end{theorem}

\begin{theorem}
    Пусть при каждом фиксированном $y \in Y$ функция $f(x, y)$ непрерывна по $X$ в точке $x_0$ и $f \rightrightarrows
    \varphi$ по $X$ при $y \rightarrow y_0$. Тогда $\varphi$ также непрерывна в точке $x_0$.
\end{theorem}

\begin{theorem}[о предельном переходе под знаком интеграла]
    Пусть функция $f: [a, b] \times Y \mapsto \mathbb{R}$ непрерывна на $[a, b]$ при каждом фиксированном $y \in Y$.
    Пусть также $f \rightrightarrows \varphi$ по $[a, b]$ при $y \rightarrow y_0$. Тогда
    \[
        \int_{a}^{b}f(x, y) dx \rightarrow \int_{a}^{b}\varphi(x)dx \text{ при } y \rightarrow y_0
    \]
\end{theorem}

\begin{proof}
    Доказательства этих теорем ничем не отличаются от аналогичных доказательств при $Y = \mathbb{N}, y_0 = +\infty$ (то
    есть для случая функциональных последовательностей)
\end{proof}

\begin{definition}
    Будем рассматривать несобственные интегралы $I(y) = \int_{a}^{b}f(x, y)dx$ с особенностью в точке $b$. То есть $f:
    [a, b) \times Y \mapsto \mathbb{R}$, причем $f$ интегрируема по Риману для любого отрезка $[a, \eta] \subset [a,
        b)$. Напомним, что символ $\int_{a}^{b}f(x, y)dx$ на самом деле означает $\lim_{\eta \rightarrow b}
        \int_{a}^{\eta}f(x, y)dx$. Говорят, что при фиксированном $y$ $I(y)$ сходится, если последний предел существует
        и конечен. Иначе $I(y)$ расходится.
\end{definition}

\begin{definition}
    Пусть несобственный интеграл $I(y)$ сходится для любого $y \in Y$. Будем говорить, что интеграл сходится равномерно,
    если
    \[
        \sup_{y \in Y} \left| \int_{\eta}^{b}f(x, y) dx \right| \rightarrow 0 \text{ при } \eta \rightarrow b - 0
    \]
\end{definition}

\begin{theorem}[критерий Коши]
    Для того, чтобы несобственный интеграл $I(y)$ сходился равномерно на $Y$ необходимо и достаточно выполнение условия
    Коши:
    \[
        \forall \varepsilon > 0: \exists \eta \in [a, b): \forall \eta', \eta'' \in [\eta, b): \sup_{y \in Y} \left|
        \int_{\eta'}^{\eta''}f(x, y)dx \right| < \varepsilon
    \]
\end{theorem}

\begin{proof}
    Докажем необходимость. Пусть $I(y)$ сходится равномерно. Тогда
    \[
        \sup_{y \in Y} \left| \int_{\eta}^{b}f(x, y)dx \right| \rightarrow 0 \text{ при } \eta \rightarrow b - 0
    \]
    То есть
    \[
        \forall \varepsilon > 0: \exists \eta \in [a, b): \forall \eta' \in (\eta, b): \sup_{y \in Y} \left| \int_{\eta'}^{b}f(x, y)dx
        \right| < \varepsilon
    \]
    Но
    \[
        \int_{\eta'}^{\eta''}f(x, y)dx = \int_{\eta'}^{b}f(x, y)dx - \int_{\eta''}^{b}f(x, y)dx
    \]
    Также мы знаем, что оба правых интеграла по модулю меньше $\varepsilon$. Но тогда необходимость доказана.

    Теперь докажем достаточность. При фиксированном $y$ выполнено:
    \[
        \left| \int_{\eta'}^{\eta''}f(x, y)dx \right| \leq \sup_{y \in Y} \left| \int_{\eta'}^{\eta''} f(x, y)dx \right|
    \]
    Но тогда выполнено
    \[
        \forall \varepsilon > 0: \exists \eta \in [a, b): \forall \eta', \eta'' \in [\eta, b): \left|
        \int_{\eta'}^{\eta''}f(x, y)dx \right| < \varepsilon
    \]
    А это критерий Коши существование предела в точке $b - 0$. Значит при любом фиксированном $y$ предел $\lim_{\eta
    \rightarrow b} \int_{a}^{b}f(x, y)dx$ существует. Теперь докажем, что интеграл сходится равномерно. При
    фиксированном $\varepsilon > 0$ существует такая $\eta \in (a, b)$, что для любых $\eta', \eta'' \in [\eta, b):
        \sup_{y \in Y} \left| \int_{\eta'}^{\eta''}f(x, y)dx \right| < \varepsilon$. Переходя к пределу в последнем неравенстве при
        $\eta'' \rightarrow b - 0$ имеем:
        \[
            \sup_{y \in Y} \left| \lim_{\eta'' \rightarrow b - 0}
            \int_{\eta'}^{\eta''}f(x, y)dx \right| < \varepsilon
        \]. То есть $\sup_{y \in Y} \left| \int_{\eta'}^{b} f(x, y)dx
        \right| < \varepsilon$. А это то, что требовалось доказать.
\end{proof}

\begin{theorem}[признак Вейерштрасса]
    Пусть $f : [a, b) \times Y \mapsto \mathbb{R}$, $\varphi: [a, b) \mapsto \mathbb{R}$. Пусть также $\forall (x, y)
    \in [a, b) \times Y: |f(x, y)| \leq \varphi(x)$ и $\int_{a}^{b}\varphi(x)dx$ сходится. Тогда
     $\int_{a}^{b}f(x, y)dx$ сходится равномерно.
\end{theorem}

\begin{proof}
    \[
        \left| \int_{\eta'}^{\eta''}f(x, y)dx \right| \leq \int_{\eta'}^{\eta''}|f(x, y)|dx \leq \int_{\eta'}^{\eta'}
        \varphi(x)dx
    \]
    Так как $\int_{a}^{b}\varphi(x)dx$ сходится, то из критерия Коши сходимости несобственного интеграла следует, что
    \[
        \forall \varepsilon > 0: \exists \eta \in [a, b): \forall \eta', \eta'' \in [\eta, b):
        \left|\int_{\eta'}^{\eta''}\varphi(x)dx \right| < \varepsilon\
    \]. Значит по предыдущей теореме интеграл $\int_{a}^{b}f(x, y)dx$ сходится равномерно.
\end{proof}

\begin{theorem}[признак сравнения]
    Пусть для функций $f, g: [a, b) \times Y \mapsto \mathbb{R}$ и некоторого $M$ выполнено, что $\forall (x, y) \in [a,
        b) \times Y: |f(x, y)| \leq Mg(x, y)$. Пусть также несобственный интеграл $\int_{a}^{b}g(x, y)dx$ сходится
        равномерно. Тогда сходится равномерно и интеграл $\int_{a}^{b}f(x, y)dx$
\end{theorem}

\begin{proof}
    Аналогично доказательству признака Вейерштрасса.
\end{proof}

\begin{theorem}
    Пусть функция $f$ непрерывна на $[a, b) \times [c, d]$ и $I(y) = \int_{a}^{b}f(x, y)dx$ сходится равномерно. Тогда
    $I(y)$ непрерывен на $[c, d]$
\end{theorem}

\begin{proof}
    Пусть $\varepsilon > 0$. Тогда $\exists \eta \in [a, b)$:
    \[
        \sup_{y \in Y} \left| \int_{\eta}^{b}f(x, y)dx \right| < \varepsilon
    \]
    Пусть $y, y + \Delta y \in [c, d]$. Тогда
    \begin{multline*}
        |I(y + \Delta y) - I(y)| \leq \int_{a}^{\eta} \left[f(x, y + \Delta y) - f(x, y) \right]dx +
        \left| \int_{\eta}^{b}f(x, y)dx \right| + \\
        \left| \int_{\eta}^{b}f(x, y + \Delta y)dx \right|
        \leq
        2 \varepsilon + \omega(\Delta y, f, Y_{\varepsilon})
    \end{multline*}
    , где $Y_{\varepsilon} := [a, \eta] \times Y$. Но колебание $\omega(\Delta y, f, Y_{\varepsilon})$ стремится к нулю.
    Значит, $I(y)$ непрерывен.
\end{proof}

\begin{theorem}[об интегрировании под знаком интеграла]
    Пусть $f$ непрерывна на $[a, b) \times [c, d]$, $I(y)$ сходится равномерно.
    Тогда $I(y)$ интегрируем на $[c, d]$, причем
    \[
        \int_{c}^{d}I(y)dy = \int_{a}^{b} \int_{c}^{d}f(x, y)dy dx
    \]
\end{theorem}

\begin{proof}
    Пусть $\eta \in (a, b)$. Поскольку $f$ непрерывна, то
    \[
        \int_{c}^{d} \int_{a}^{\eta}f(x, y)dxdy = \int_{a}^{\eta} \int_{c}^{d}f(x, y)dydx
    \]
    Перейдем к пределу при $\eta \rightarrow b - 0$. Тогда правая часть стремится к $\int_{c}^{d}I(y)dy$. Докажем это:
    \[
        \left| \int_{c}^{d} \int_{a}^{b} f(x, y)dxdy - \int_{c}^{d} \int_{a}^{\eta}f(x, y)dxdy \right| \leq (d - c)
        \sup_{y \in Y} \left| \int_{\eta}^{b}f(x, y)dx \right|
    \]
    Последнее выражение стремится к нулю при $\eta \rightarrow b - 0$. Значит
    \[
        \exists \lim_{\eta \rightarrow b - 0} \int_{a}^{\eta} \int_{c}^{d}f(x, y)dydx = \int_{c}^{d}I(y)dy
    \]
\end{proof}

\begin{theorem}[о дифференцировании под знаком интеграла]
    Пусть $f, f'_y$ непрерывны на $[a, b) \times [c, d]$. Пусть для некоторого $y_0 \in [c, d]$ сходится интеграл
    $I(y_0) = \int_{a}^{b}f(x, y_0)dx$, а интеграл $\int_{a}^{b}f'_y(x, y)dx$ сходится равномерно на $[c, d]$. Тогда
    $I(y)$ дифференцируема, причем
    \[
        I'(y) = \int_{a}^{b}f'_y(x, y)dx
    \]
\end{theorem}

\begin{proof}
    В силу предыдущей теоремы:
    \begin{multline*}
        \int_{y_0}^{y} \int_{a}^{b} f'_y(x, y)dxdy = \int_{a}^{b} \int_{y_0}^{y} f'_y(x, y)dydx = \\
        \int_{a}^{b} \left[ f(x, y) - f(x, y_0) \right] dx = \int_{a}^{b}f(x, y)dx - \int_{a}^{b}f(x, y_0)dx
    \end{multline*}
    Дифференцируя полученное тождество, имеем:
    \[
        \int_{a}^{b}f'_y(x, y)dx = \frac{d}{dy} \int_{a}^{b}f(x, y)dx
    \]
\end{proof}

\begin{theorem}[признак Дирихле]
    Пусть функции $f, g: [a, +\infty) \times Y \mapsto \mathbb{R}$ таковы, что при любом фиксированном $y \in Y$ функции 
    $f, g'_x$ непрерывны, а $g$ монотонно убывает, причём:
    \begin{enumerate}
        \item $\exists M: \forall y \in Y: \forall b > a: \left| \int_{a}^{b}f(x, y)dx \right| \leq M$, то есть
            интегралы $\int_{a}^{b}f(x, y)dx$ равномерно ограничены на $Y$
        \item $g(x, y) \rightrightarrows 0$ по $Y$ при $x \rightarrow +\infty$, то есть $\forall \varepsilon > 0: \exists 
            b > a: \forall x > b: \forall y: |g(x, y)| < \varepsilon$
    \end{enumerate}
    Тогда интеграл
    \[
        \int_{a}^{+\infty}f(x, y)g(x, y)dx
    \] сходится равномерно.
\end{theorem}

\begin{proof}
    Обозначим $F(x, y) = \int_{a}^{x}f(t, y)dt$. Тогда при $a < b < c:$
    \begin{multline*}
        \left| \int_{b}^{c} f(x, y) g(x, y)dx \right| = \left| \int_{b}^{c}g(x, y) dF(x, y) \right| = \\ \left| F(x,
        y) g(x, y) \Big|_b^c - \int_{b}^{c}F(x, y) g'_x(x, y)dx \right| \leq \\ 2M|g(b, y)| + \left|
        \int_{b}^{c}F(x, y) g'_x(x, y)dx \right| \leq 2M|g(b, y)| + \\ \int_{b}^{c}|F(x, y)||g'_x(x, y)|dx \leq
        2M|g(b, y)| + \int_{b}^{c}M|g'(x, y)|dx = 2M|g(b, y)| + \\ M \int_{b}^{c}|g'_x(x, y)|dx = 2M|g(b, y)| + M
        \left|\int_{b}^{c}g'_x(x, y)dx \right| = 2M|g(b, y)| + \\ M \left| g(c, y) - g(b, y) \right| \leq 4M |g(b, y)| <
        4M \varepsilon
    \end{multline*}
    Значит по критерию Коши интеграл сходится равномерно.
\end{proof}

\end{document}
