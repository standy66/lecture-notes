\documentclass[document.tex]{subfiles}

\begin{document}

\section{Интегралы, зависящие от параметра}

\begin{definition}
    Интегралы Римана вида
    \[
        I(y) = \int_{a}^{b}f(x, y)dx
    \]
    \[
        J(y) = \int_{\varphi(y)}^{\psi(y)}f(x, y)dx
    \]
    называются интегралами, зависящими от параметра.
\end{definition}

\begin{theorem}
    Если функция $f(x, y)$ непрерывна на $[a, b] \times [c, d]$, то интеграл $I(y) = \int_{a}^{b}f(x, y)dx$ непрерывен
    на $[c, d]$ как функция от $y$
\end{theorem}

\begin{proof}
    Пусть $y_1, y_2 \in [c, d]$. Обозначим $\Delta y = y_2 - y_1$. Тогда
    \begin{multline*}
        |I(y_1) - I(y_2)| = \left| \int_{a}^{b}(f(x, y_1) - f(x, y_2))dx \right| \leq \int_{a}^{b}|f(x, y_1) - f(x,
        y_2)|dx \leq \\
        (b - a) \max_{[a, b]} f(x, y_1) - f(x, y_2) \leq (b - a) \omega(|\Delta y|, f)
    \end{multline*}
    Поскольку $f$ непрерывна на компакте, а следовательно, и равномерно непрерывна на нём, то колебание $\omega(\delta,
    f) \rightarrow 0$ при $\delta \rightarrow 0$.
\end{proof}

\begin{theorem}
    Пусть $\varphi, \psi : [c, d] \mapsto \mathbb{R}$ -- непрерывные на $[c, d]$ функции, причем $\varphi \leq \psi$.
    Пусть $f$ непрерывна на $\overline G := \{(x, y) \in \mathbb{R}^2 : \varphi(y) \leq x \leq \psi(y), c \leq y \leq d\}$.
    Тогда интеграл $J(y) = \int_{\varphi(y)}^{\psi(y)}f(x, y)dx$ непрерывен на $[c, d]$
\end{theorem}

\begin{proof}
    Осуществим замену переменных: $x = t \psi(y) + (1 - t) \varphi(y)$. Тогда $dx = (\psi(y) - \varphi(y))dt$. Тогда
    $J(y) = \int_{0}^{1}f(t \psi(y) + (1 - t) \varphi(y), y)(\psi(y) - \varphi(y))dt$. Функция $g(t, y) = f(t \psi(y) +
    (1 - t) \varphi(y), y) (\psi(y) - \varphi(y))$ является непрерывной на $[c, d]$ по теореме о композиции непрерывных
    функций. Значит интеграл $J(y)$ непрерывен на $[c, d]$.
\end{proof}

\begin{theorem}
    Пусть для интеграла $I(y) = \int_{a}^{b}f(x, y)dx$ выполнены условия:
    \begin{enumerate}
        \item $f(x, y)$ интегрируема на $[a, b] \times [c, d]$
        \item $\forall y \in [c, d]$ существует интеграл $I(y)$
        \item $\forall x \in [a, b]$ существует интеграл $\int_{c}^{d}f(x, y) dy$
    \end{enumerate}
    Тогда существуют оба повторных интеграла и выполнено равенство:
    \[
        \int_{a}^{b} \int_{c}^{d} f(x, y) dxdy = \int_{c}^{d} \int_{a}^{b} f(x, y) dx dy
    \]
\end{theorem}

\begin{proof}
    Следует из соответствующей теоремы о повторных интегралах.
\end{proof}

\begin{corollary}
    Если $f$ непрерывна на $[a, b] \times [c, d]$, то вышесформулированная теорема также справедлива.
\end{corollary}

\begin{theorem}[правило Лейбница]
    Пусть $f, \frac{\partial f}{\partial y}$ непрерывны на $[a, b] \times [c, d]$. Тогда интеграл
    $I(y) = \int_{a}^{b}f(x, y)dx$ дифференцируем, причем
    \[
        \frac{dI(y)}{dy} = \int_{a}^{b}\frac{\partial f}{\partial y}(x, y) dx
    \]
\end{theorem}

\begin{proof}
    Рассмотрим $y, y + \Delta y \in [c, d]$. Воспользовавшись формулой конечных приращений Лагранжа для некоторого
    $\theta \in (0, 1)$ выполнено:
    \begin{multline*}
        \left|\frac{I(y + \Delta y) - I(y)}{\Delta y} - \int_{a}^{b}\frac{\partial}{\partial y}f(x, y)dx \right| = \\
        \left| \int_{a}^{b} \left[ \frac{f(x, y + \Delta y) - f(x, y)}{\Delta y} - \frac{\partial}{\partial y}f(x, y)
        \right] dx\right| \leq \\
        \int_{a}^{b} \left|\frac{\partial}{\partial y}f(x, y + \theta \Delta y) - \frac{\partial}{\partial y}f(x, y)
        \right| dx \leq  (b - a) \omega(\Delta y, \frac{\partial f}{\partial y})
    \end{multline*}
    Поскольку $\frac{\partial f}{\partial y}$ непрерывно на $[a, b] \times [c, d]$, а следовательно, и равномерно
    непрерывно, $\omega(\Delta y, \frac{\partial f}{\partial y}) \rightarrow 0$ при $\Delta y \rightarrow 0$
\end{proof}

\begin{theorem}
    Пусть $f, f'_y$ непрерывны на
    $\overline G := \{(x, y) \in \mathbb{R}^2 : \varphi(y) \leq x \leq \psi(y), c \leq y \leq d\}$.
    Функции $\varphi, \psi$ непрерывно дифференцируемы на $[c, d]$. Тогда $J(y)$ также дифференцируем на $[c, d]$,
    причем
    \[
        J'(y) = f(\psi(y), y)\psi'(y) - f(\varphi(y), y) \varphi'(y) + \int_{\varphi(y)}^{\psi(y)}f'_y(x, y)dx
    \]

\end{theorem}

\begin{proof}
    Рассмотрим $F(y, u, v) = \int_{u}^{v}f(x, y) dx$. Тогда $J(y) = F(y, \varphi(y), \psi(y))$. По правилу
    дифференцирования сложной функции $J'_y = F'_y + F'_u \varphi'(y) + F'_v \psi'(y)$.
    \[
        F'_y = \int_{u}^{v}f'_y(x, y)dx
    \] по правилу Лейбница. Применяя теорему о среднем для интеграла, находим, что $F'_v$  в точке $v_0$ это:
    \[
        \left( \int_{u}^{v}f(x, y)dx \right)'_v = \lim_{v \rightarrow v_0} \frac{\int_{v}^{v_0}f(x, y)dx}{v - v_0} =
        \lim_{v \rightarrow v_0} f(\xi, y) = f(v_0, y)
    \]
    Поэтому
    \[
        F'_u = -f(u, y)
    \]
    \[
        F'_v = f(v, y)
    \] Заметим, что $F'_u$ и $F'_v$ непрерывны, так как $f$ непрерывна, а $F'_y$ также непрерывна. Чтобы доказать это,
    достаточно сделать замену $x = t v + (1 - t)u$:
    \[
        F'_y = \int_{0}^{1}f(tv + (1 - t)u, y)(v - u) dt
    \]
    Подинтегральная функция непрерывна, поэтому и $F'_y$ непрерывна (доказывается так же, как и первая теорема в этом
    разделе)
\end{proof}

\end{document}
