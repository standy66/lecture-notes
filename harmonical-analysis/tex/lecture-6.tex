\documentclass[document.tex]{subfiles}

\begin{document}
\section{Гильбертовы пространства}
\begin{definition}
    Бесконечномерное евклидово пространство называется предгильбертовым
\end{definition}

\begin{definition}
    Полное предгильбертово пространство называется гильбертовым.
\end{definition}

\begin{statement}
    В предгильбертовых пространствах c нормой $\|x\| = \sqrt{(x, x)}$ скалярное произведение $(x, y)$ непрерывно по $x$ и $y$
\end{statement}

\begin{proof}
    При $\|\Delta x\| + \| \Delta y\| = o(1)$ (а это эквивалентно тому, что $\|\Delta x\| = o(1), \|\Delta
    y\| = o(1)$, ведь $0 \leq \|\Delta x\| \leq \|\Delta x \| + \| \Delta y\| = o(1)$) верно, что
    \begin{multline*}
        0 \leq |(x + \Delta x, y + \Delta y) - (x, y)| = |(x, y) + (x, \Delta y) + (\Delta x, y) + (\Delta x, \Delta y) - (x,
        y)| = \\
        |(x, \Delta y) + (\Delta x, y) + (\Delta x, \Delta y)| \leq |(x, \Delta y)| + |(\Delta x, y)| + |(\Delta x,
        \Delta y)| \leq \\
        \| x \| \|\Delta y\| + \|\Delta x\| \|y\| + \|\Delta x\| \|\Delta y\| = o(1)
    \end{multline*}
\end{proof}

\begin{corollary}
    Так как $(x, y)$ непрерывно, то верно следующее:
    \begin{enumerate}
        \item Если $x_k \rightarrow x$, то $(x_k, a) \rightarrow (x, a)$
        \item Если $\sum_{i = 1}^{\infty} x_k = x$, то $\sum_{i = 1}^{\infty} (x_k, a) = (x, a)$
    \end{enumerate}
\end{corollary}

\subsection{Ортогональные системы}
\begin{definition}
    Элементы $a, b$ называются ортогональными, если $(a, b) = 0$
\end{definition}

\begin{definition}
    Система из ненулевых элементов $\{e_i\}_{i = 1}^{\infty}$ называется ортогональной, если её элементы попарно ортогональны.
\end{definition}

\begin{statement}
    Любая нетривиальная конечная линейная комбинация элементов ортогональной системы не равна нулю
\end{statement}

\begin{proof}
    Пусть $\exists N : \exists (\lambda_1, \cdots, \lambda_N) \neq 0: \sum_{i = 1}^{N} \lambda_i e_i = 0$

    Выберем $s \in \{1, \cdots, N\}$. Тогда умножив линейную комбинацию скалярно на $e_s$ получим:
    $\sum_{i = 1}^{N} \lambda_i (e_i, e_s) = 0$. Значит, $\lambda_s (e_s, e_s) = 0$. Но $(e_s, e_s) \neq 0$. А значит,
    $\lambda_s = 0$. И так для любого $s$. Противоречие.
\end{proof}

\begin{theorem}
    Пусть $x = \sum_{i = 1}^{\infty} \alpha_i e_i$, где $\{e_i\}_{i = 1}^{\infty}$ -- ортогональная система.
    Тогда $\alpha_i = \frac{(x, e_i)}{(e_i, e_i)}$ ($\alpha_i$ называются коэффициентами Фурье по системе $\{e_i\}_{i =
    1}^{\infty}$)
\end{theorem}

\begin{theorem}
    $(x, e_s) = \sum_{i = 1}^{\infty} \alpha_i (e_i, e_s) = \alpha_s (e_s, e_s)$. Значит, $\alpha_s =
    \frac{(x, e_s)}{(e_s, e_s)}$
\end{theorem}

\begin{remark}
    Обозначение $x \sim \sum_{i = 1}^{\infty} \alpha_i e_i$ означает, что $\alpha_i$ -- коэффициенты Фурье элемента $x$
    по системе $e_i$
\end{remark}

\begin{theorem}[Минимальное свойство коэффициентов Фурье]
    Пусть $\{e_i\}_{i = 1}^{\infty}$ -- ортогональная система в предгильбертовом пространстве $H$. $x \in H$, $\alpha_i$
    -- коэффициенты Фурье элемента $x$ по системе $\{e_i\}_{i = 1}^{\infty}$. Пусть $n \in \mathbb{N}$. Тогда $\forall
    c_1, \cdots, c_n: \|x - \sum_{i = 1}^n c_i e_i\| \geq \|x - \sum_{i = 1}^n \alpha_i e_i\|$
\end{theorem}

\begin{proof}
    Зафиксируем $c_1, \cdots, c_n$. Заметим, что $(x, e_i) = \alpha_i \|e_i\|^2$.
    Рассмотрим цепочку эквивалентных преобразований:
    \begin{multline*}\\
        \|x - \sum_{i = 1}^n c_i e_i\| \geq \|x - \sum_{i = 1}^n \alpha_i e_i\| \\
        (x - \sum_{i = 1}^n c_i e_i, x - \sum_{i = 1}^n c_i e_i) \geq (x - \sum_{i = 1}^n \alpha_i e_i, x - \sum_{i =
        1}^n \alpha_i e_i) \\
        (x, x) - 2 \sum_{i = 1}^n c_i (x, e_i) + \sum_{i = 1}^n c_i^2 \|e_i\|^2 \geq (x, x) - 2 \sum_{i = 1}^n \alpha_i
        (x, e_i) + \sum_{i = 1}^n \alpha_i^2 \|e_i\|^2 \\
        \sum_{i = 1}^n c_i^2 \|e_i\|^2 - 2 \sum_{i = 1}^n c_i \alpha_i \|e_i\|^2 \geq \sum_{i = 1}^n \alpha_i^2
        \|e_i\|^2 - 2 \sum_{i = 1}^n \alpha_i^2 \|e_i\|^2 \\
        \sum_{i = 1}^n \left( c_i^2 \|e_i\|^2 - 2 c_i \alpha_i \|e_i\|^2 + \alpha_i^2 \|e_i\|^2 \right) \geq 0\\
        \sum_{i = 1}^n \|e_i\|^2 (c_i - \alpha_i)^2 \geq 0\\
    \end{multline*}
\end{proof}

\begin{theorem}
    Пусть $c = a + b$ и $a$ ортогонально с $b$. Тогда $\|c\|^2 = \|a\|^2 + \|b\|^2$
\end{theorem}

\begin{proof}
    $\|c\|^2 = (c, c) = (a + b, a + b) = (a, a) + 2 (a, b) + (b, b) = (a, a) + (b, b) = \|a\|^2 + \|b\|^2$
\end{proof}

\begin{statement}
    Пусть $x \sim \sum_{i = 1}^{\infty} \alpha_i e_i$. Обозначим $S_n = \sum_{i = 1}^{n} \alpha_i e_i$. Тогда $\forall k
    > n: (S_n, e_k) = 0$
\end{statement}

\begin{lemma}[об ортогональном разложении]
    Любой элемент $x$ может быть представлен как сумма двух ортогональных слагаемых:
    \[
        x = S_n + (x - S_n)
    \], причем $(S_n, x - S_n) = 0$. В частности, $\|x\|^2 = \|S_n\|^2 + \|x - S_n\|^2$
\end{lemma}

\begin{proof}
    \[
        (S_n, x - S_n) = (S_n, x) - (S_n, S_n) = \sum_{i = 1}^n \alpha_i (x, e_i) - \sum_{i = 1}^n \alpha_i^2 \|e_i\|^2 =
        0
    \]
\end{proof}

\begin{theorem}[неравенство Бесселя]
    Пусть $\{e_i\}_{i = 1}^{\infty}$ -- ортогональная система, $x \sim \sum_{i = 1}^{\infty} \alpha_i e_i$. Тогда
    $\|x\|^2 \geq \sum_{i = 1}^{\infty} \alpha_i^2 \|e_i\|^2$
\end{theorem}

\begin{proof}
    $\|x\|^2 = \|S_n\|^2 + \|x - S_n\|^2 \geq \|S_n\|^2 = \sum_{i = 1}^n \alpha_i^2 \|e_i\|^2$. Заметим, что это
    выполено для любого $n$. Тогда переходя к пределу $n \rightarrow \infty$ получаем, что $\|x\|^2 \geq \sum_{i =
    1}^{\infty} \alpha_i^2 \|e_i\|^2$
\end{proof}

\begin{theorem}

\end{theorem}

\end{document}

